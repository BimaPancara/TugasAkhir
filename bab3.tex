%-----------------------------------------------------------------------------%
\chapter{\babTiga}
%-----------------------------------------------------------------------------%

%-----------------------------------------------------------------------------%
\section{Alur Penelitian}
%-----------------------------------------------------------------------------%
Dalam suatu penelitian, terdapat urutan tahapan yang perlu dilakukan. Alur penelitian ini mengandung seluruh langkah yang harus ditempuh, mulai dari fase perancangan hingga tahap akhir penelitian.
 \begin{figure}
	\begin{center}
		\includegraphics[scale=0.65]{pics/bab3/FlowchartSempro.png} 
		\label{img:flowchart}
		\caption[\textit{Flowchart} Penelitian]{\textit{Flowchart} Penelitian}
	\end{center}
\end{figure}
Pada alur penelitian yang telah dirancang, terdapat beberapa tahap yang perlu dilakukan setelah penelitian dimulai dan sebelum penelitian diakhiri. Tiap tahapan yang telah dirancang harus dilaksanakan sebaik mungkin agar hasil yang diharapkan dapat tercapai.
	
\section{Penentuan Parameter}

Pada tahap ini parameter pengujian ditentukan sehingga hasil yang dicapai dapat dikatakan baik, sebagai berikut.

\begin{center}
	\begin{longtable}{| c | c | c |}
		\caption{Parameter Pengujian}
		\label{tab:paramUji}\\
		\hline
		No. & Parameter Pengujian		& Satuan\\ \hline
		1.  &Jarak	   					& m\\
		2.  &Kecepatan 					& m/s\\
		3.  &\textit{RMSE}				& -\\
		4.	&Nilai prediksi \textit{beat frequency}	& Hz \\
		5.	&Nilai prediksi \textit{doppler frequency shift} & Hz \\
		\hline
	\end{longtable}
\end{center}

\section{Perancangan Spesifikasi Sistem}
Pada tahap ini, dilakukan perancangan tentang penelitian yang diangkat, dalam konteks ini adalah radar. Sehingga perlu dilakukannya penentuan spesifikasi radar berdasarkan perangkat keras yang digunakan. Penelitian ini menggunakan alat USRP berseri B210.  Spesifikasi dari alat ini akan dijelaskan pada tabel berikut.

\begin{center}
	\begin{longtable}{| c | c | c |}
		\caption{Spesifikasi Sistem Radar}
		\label{tab:spekRadar}\\
		\hline
		No. & Spesifikasi 					& Keterangan\\\hline
		1.  & USRP 							& B210\\
		2.  & \textit{Center Frequency}  	& 3000 MHz \\
		3.  & \textit{Bandwidth} 			& 50 MHz \\
		4.	& Bentuk Modulasi				& \textit{Triangular}\\
		5.  & Jarak Maksimum 				& 150 km \\
		6.  & Resolusi Jarak 				& 3 m \\
		7.  & Kecepatan Maksimum			& 15 $m/s$ \\
		8.  & Resolusi Kecepatan 			& 1 $m/s$\\
		9.	& Durasi \textit{Chirp}			& 0.001667 s\\
		10.	& \textit{Chirp Rate}			& 30000 MHz\\
		\hline
	\end{longtable}
\end{center}

\begin{itemize}
	\item Hitung panjang gelombang ($\lambda$) dari frekuensi tengah yang sudah ditentukan yaitu 3 GHz.
	\begin{align*}
		\lambda &= \frac{c}{F_{c}}\\
		\lambda &= \frac{3 \cdot 10^{8}}{3 \cdot 10^{9}}\\
		\lambda &= 0.1 m
	\end{align*}

	\item Menghitung resolusi jarak berdasarkan persamaan \ref{eq:RangeRes} dan dengan menentukan \textit{bandwidth} bernilai 50 MHz, maka.
		\begin{align*}
			R_{res} &= \frac{c}{2 BW} \\
			R_{res} &= \frac{3 \cdot 10^{8}}{2 \cdot 50 MHz}\\
			R_{res} &= 3 m
		\end{align*}
	\item Menghitung jarak maksimum yang dapat dideteksi oleh radar digunakanlah persamaan \ref{eq:MaxRange}, namun sebelumnya harus ditentukan terlebih dahulu nilai $\mu$, yang merupakan tingkat kenaikan frekuensi pada suatu periode sesuai dengan persamaan \ref{eq:chirpRate}, dengan nilai $T_{c}$ sesuai persamaan \ref{eq:chirpTime} dan nilai kecepatan maksimum ditentukan bernilai 15 m/s, maka.
	
	\begin{align*}
		T_{c} &= \frac{\lambda}{4 \cdot V_{max}}\\
		T_{c} &= \frac{0.1}{4 \cdot 15}\\
		T_{c} &= 0.001667
	\end{align*}

	\item 
	Sehingga nilai $\mu$ dapat dihitung menjadi.

		\begin{align*}
		\mu &= \frac{\textit{Bandwidth}}{T_{c}}\\
		\mu &= \frac{\textit{50 MHz}}{0.001667}\\
		\mu &= 30000 MHz/s
		\end{align*}

	\item 	
	Dengan jarak maksimum yang didapat adalah.
		\begin{align*}
		R_{max} &= \frac{F_{s} \cdot c}{2 \cdot \mu}\\
		R_{max} &= \frac{30 \cdot 10^{6} \cdot 3 \cdot 10^{8}}{2 \cdot 30000}\\
		R_{max} &= 150 m
		\end{align*}

	\item 
	Dengan $T_{f}$ sebagai durasi \textit{frame} bernilai 0.05 s maka resolusi kecepatannya.
		\begin{align*}
			V_{res} &= \frac{\lambda}{2 \cdot T_{f}}\\
			V_{res} &= \frac{0.1}{2 \cdot 0.05}\\
			V_{res} &= 1 m/s
		\end{align*}

\end{itemize}

\section{Implementasi Sistem}
Tahap implementasi ini dilakukan pada aplikasi GNURadio dan menghasilkan \textit{flow diagram} yang merepresentasikan langkah yang dilakukan pada USRP. \textit{Flow diagram} yang didesain sudah memenuhi spesifikasi sistem radar pada tabel \ref{tab:spekRadar}. 

Implementasi sistem akan dilaksanakan pada beberapa perangkat, mulai dari laptop, antena, dan USRP. Berikut detail perangkat yang akan digunakan pada saat implementasi guna mendapat hasil yang baik.

\begin{enumerate}
	\item \textit{IdeaPad Gaming 3 15ARH7} :
	\begin{figure}
		\begin{center}
			\includegraphics[scale=0.2]{pics/bab3/laptop.jpg} 
			\caption[Gambar Perangkat Laptop Yang Digunakan]{Gambar Perangkat Laptop Yang Digunakan}
			\label{pic:contohBlokGRC}
		\end{center}
	\end{figure}

	\begin{itemize}
		\item \textit{Processor} : AMD Ryzen 7 6800H dengan \textit{Radeon Graphics} 3.20 GHz
		\item \textit{Memory} : 8,00 GB (7,19 GB \textit{usable})
	\end{itemize}

	\item Perangkat \textit{Software Defined Radio} :
	\begin{figure}
		\begin{center}
			\includegraphics[scale=0.045]{pics/bab3/usrp2.jpg}
			\caption{Alat USRP B210}
			\label{img:logPeriodic}
		\end{center}
	\end{figure}
	\begin{itemize}
		\item Tipe : USRP B210 
		\item Jarak Frekuensi : 70 MHz - 6000 MHz 
	\end{itemize}

	\item Antena \textit{Log-periodic} :
	\begin{figure}
		\begin{center}
			\includegraphics[scale=0.05]{pics/bab3/logPeriodic.jpg}
			\caption{Antena \textit{Log Periodic} Pengujian}
			\label{img:usrpBoard}
		\end{center}
	\end{figure}
	\begin{itemize}
		\item Frekuensi : 800 MHz - 6000 MHz 
		\item Pola Radiasi : \textit{Directional}
		\item \textit{Gain} : 5.2 - 6.3 dB
	\end{itemize}
\end{enumerate}

	
\section{Pengambilan Data}
Pada tahap ini, pengambilan data dengan radar yang sudah didesain dan diimplementasikan pada USRP dilakukan. Pengujian dilakukan dengan menggunakan kendaraan roda empat sebagai objek yang akan dideteksi. Sehingga pengambilan data kecepatan dan prediksi jarak dapat dilakukan. Hasil prediksi jarak dan kecepatan radar akan dibandingkan dengan nilai aktual jarak pada kenyataan dan kecepatan tercatat pada \textit{speedometer}.

\begin{figure}
	\begin{center}
		\includegraphics[scale=0.55]{pics/bab3/skema.png}
		\caption{Skema Penelitian}
		\label{img:skema}
	\end{center}
\end{figure}

Data berupa nilai \textit{beat frequency} dan \textit{doppler frequency shift} yang sudah ditentukan sebagai parameter pengujian telah didapat dari hasil pengambilan data akan dibandingkan dengan nilai prediksi berdasarkan perhitungan. Dengan begitu, maka nilai RMSE dapat dihitung.

\begin{figure}
	\begin{center}
		\includegraphics[scale=0.35]{pics/bab3/petaPengujian.png}
		\caption{Lokasi Pengujian}
		\label{img:petaUji}
	\end{center}
\end{figure}

Pengambilan data akan dilaksanakan di lokasi lapangan Univertitas Telkom Surabaya yang beralamat Jl. Ketintang No.156, Ketintang, Kec. Gayungan, Surabaya, Jawa Timur 60231.

\section{Konfigurasi Pengujian}
Konfigurasi pengujian dilakukan sesuai dengan gambar \ref{img:skema}. Terdapat satu buah perangkat laptop yang terhubung dengan dua buah USRP, masing-masing USRP terhubung dengan antena \textit{Log-periodic}. USRP 1 berperan sebagai \textit{transmitter} sedangkan USRP 2 berperan sebagai \textit{receiver}.

\begin{figure}
	\begin{center}
		\includegraphics[scale=0.09]{pics/bab3/konfigurasiPengujian.jpg}
		\caption{Konfigurasi Pengujian}
		\label{img:konfigurasi}
	\end{center}
\end{figure}


% \section{Perhitungan Nilai Simulasi}
% \todo{Hitung Nilai $F_{b}$ dan $F_{d}$ dengan tabel}
% Dolore velit amet amet aliqua exercitation velit nulla ad eu. Ad voluptate minim fugiat et mollit commodo elit. Excepteur minim magna aute commodo consectetur velit aute et consectetur sit. Fugiat voluptate ea officia labore. Ut aute voluptate proident officia consequat nostrud aute nulla consequat enim. Lorem nostrud sit pariatur officia Lorem officia. Aliquip labore Lorem quis ea proident reprehenderit labore sit.