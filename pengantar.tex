%-----------------------------------------------------------------------------%
\chapter*{\kataPengantar}
%-----------------------------------------------------------------------------%

Segala puji dan syukur kehadirat Allah SWT atas berkah, rahmat dan hidayah-Nya yang senantiasa dilimpahkan kepada penulis, sehingga bisa menyelesaikan Tugas Akhir dengan judul “Rancang Bangun Sistem \textit{FMCW} Radar Berbasis \textit{Software Defined Radio} dengan \textit{GNURadio} Untuk Mendeteksi, Estimasi Jarak, dan Kecepatan Objek” sebagai salah satu persyaratan dalam menyelesaikan pendidikan pada Program Studi Teknik Telekomunikasi Fakultas Teknik Elektro Telkom University Surabaya.
Penulis menyadari bahwa penulisan ini tidak dapat terselesaikan tanpa dukungan dari berbagai pihak baik moril maupun materil. Oleh karena itu, penulis ingin menyampaikan ucapan terima kasih kepada semua pihak yang telah membantu dalam penyusunan skripsi ini terutama kepada:
\begin{enumerate}
    \item Papa Ir. Mardi Susanto Hariyono, Mama Indrianingrum, S.H., dan Kakak Noviyana Haryono Putri, S.E. yang setia serta selalu mendukung keperluan penulis dari berbagai macam hal, mulai dari moril, materil, hingga pengambilan data dalam tugas akhir ini.
    \item Bapak Dr. Fannush Shofi Akbar, S.ST., selaku Ketua Program Studi Teknik Telekomunikasi Fakultas Teknik Elektro Telkom University Surabaya, sekaligus sebagai dosen Pembimbing I Tugas Akhir yang telah bersedia membimbing dan mengarahkan penulis selama menyusun Tugas Akhir dan memberikan banyak ilmu serta solusi pada setiap permasalahan atas kesulitan dalam penulisan Tugas Akhir ini.
    \item Ibu Risdilah Mimma Untsa, S.ST., M.T., selaku dosen Pembimbing II Tugas Akhir yang telah berkenan memberikan tambahan ilmu dan solusi pada setiap permasalahan atas kesulitan dalam penulisan Tugas Akhir ini.
    \item Ibu Nilla Rachmaningrum, S.T., M.T., selaku dosen Penguji I Tugas Akhir yang telah memberikan masukan dan kritik saran agar penulisan buku Tugas Akhir ini menjadi lebih baik.
    \item Bapak Muhsin, S.T., M.T., Ph.D., selaku dosen Penguji II Tugas Akhir yang telah memberikan masukan dan kritik saran agar penulisan buku Tugas Akhir ini menjadi lebih baik.
    \item Bapak Walid Maulana H, S.T., M.T., selaku Dosen Wali, yang selalu memberikan nasihat dan pengarahan selama penulis menempuh studi di Program Studi Teknik Telekomunikasi Fakultas Teknik Elektro Telkom University Surabaya.
    \item Terima kasih untuk Athallah Rafi Andro Mulyawan, Hidayat, Alifvian Aria Sasongko, Akbar Handinegara, dan Vian Akbar Putra Widodo yang sudah membantu penulis untuk mengambil data jarak serta kecepatan dalam waktu singkat dan tanpa rencana di kampus.
    \item Terima kasih untuk teman teman seperjuangan laboratorium \textit{Wireless Communication Signal Processing} Wildan Abdillah, Wahyu Agung, Arya Prayoga, Haidar Daniel, Anas Al Hibrizi, Adam Maulana, Wafiqoh Dwi, Elsa Diah, First Ad'ha, Regina Salsabilla, Qaanitah Salwa, Intan Anggreita, Vidira Anindita yang selalu saling mendukung dikala pusing dan tak tahu arah.
    \item Teman teman program studi Teknik Telekomunikasi angkatan 2021 yang tidak bisa disebutkan satu persatu, tetap semangat.
\end{enumerate}

Penulis menyadari bahwa Tugas Akhir ini masih jauh dari sempurna dikarenakan terbatasnya pengalaman dan pengetahuan yang dimiliki penulis. Akhir kata, semoga Tugas Akhir ini dapat bermanfaat dan semoga Allah SWT memberi lindungan bagi kita semua.

 
\vspace*{0.1cm}
\begin{flushright}
Surabaya, 05 Februari 2025\\[0.1cm]
\vspace*{1cm}
\penulis

\end{flushright}