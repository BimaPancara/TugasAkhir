% Tipe dokumen adalah report dengan satu kolom kertas A4 satu sisi.
% Ukuran font 12pt
\documentclass[12pt, a4, onecolumn, oneside, final]{report}

% Load konfigurasi LaTeX untuk tipe laporan thesis
\usepackage{ta}	

\usepackage{enumitem}
%\setlist[enumerate]{itemsep=1pt, topsep=1pt}

\titleformat{\section}
  {\normalfont\fontsize{12}{15}\bfseries}{\thesection}{1em}{}

\titleformat{\subsection}
  {\normalfont\fontsize{12}{15}\bfseries}{\thesubsection}{1em}{}

% Load konfigurasi khusus untuk laporan yang sedang dibuat
%-----------------------------------------------------------------------------%
% Informasi Mengenai Dokumen
%-----------------------------------------------------------------------------%
% 
% Judul laporan. 
\var{\judul}{Rancang Bangun Sistem \textit{FMCW} Radar Berbasis \textit{Software Defined Radio} dengan \textit{GNURadio} Untuk Mendeteksi, Estimasi Jarak, dan Kecepatan Objek}
% 
% Tulis kembali judul laporan, kali ini akan diubah menjadi huruf kapital
\Var{\Judul}{Rancang Bangun Sistem \textit{FMCW} Radar Berbasis \textit{Software Defined Radio} dengan \textit{GNURadio} Untuk Mendeteksi, Estimasi Jarak, dan Kecepatan Objek}
% 
% Tulis kembali judul laporan namun dengan bahasa Inggris
\var{\judulInggris}{Design of FMCW Radar Based on Software Defined Radio with GNURadio for Detection, Range Estimation, and Velocity of an Object}

\Var{\JudulInggris}{Design of FMCW Radar Based on Software Defined Radio with GNURadio for Detection, Range Estimation, and Velocity of an Object}

% 
% Tipe laporan, dapat berisi Skripsi, Tugas Akhir, Thesis, atau Disertasi
\var{\type}{Tugas Akhir}
% 
% Tulis kembali tipe laporan, kali ini akan diubah menjadi huruf kapital
\Var{\Type}{Tugas Akhir}
% 
% Tulis nama penulis 
\var{\penulis}{Bima Pancara Haryono Putra}
\var{\alamat}{Perum. Mega Asri, B-02}
\var{\tlp}{089699444339}
\var{\email}{bimapancara@student.telkomuniversity.ac.id}
% 
% Tulis kembali nama penulis, kali ini akan diubah menjadi huruf kapital
\Var{\Penulis}{Bima Pancara Haryono Putra}
% 
% Tulis NPM penulis
\var{\nim}{1101210528}
% 
% Tuliskan Kampus dimana penulis berada
\var{\kampus}{Surabaya}
\Var{\Kampus}{Surabaya}
% Tuliskan Fakultas dimana penulis berada
\Var{\Fakultas}{Fakultas Teknik Elektro}
\var{\fakultas}{Fakultas Teknik Elektro}
% 
% Tuliskan Program Studi yang diambil penulis
\Var{\Program}{Teknik Telekomunikasi}
\var{\program}{Teknik Telekomunikasi}
% 
% Tuliskan tahun publikasi laporan
\Var{\Tahun}{2025}
% 
% Tuliskan gelar yang akan diperoleh dengan menyerahkan laporan ini
\var{\gelar}{Sarjana Teknik}
% 
% Tuliskan tanggal pengesahan laporan, waktu dimana laporan diserahkan ke 
% penguji/sekretariat
\var{\tanggalPengesahan}{Januari 2025} 
% 
% Tuliskan tanggal keputusan sidang dikeluarkan dan penulis dinyatakan 
% lulus/tidak lulus
%\var{\tanggalLulus}{25 April 2015}
% 
% Tuliskan pembimbing 
\var{\pembimbingSatu}{Dr. Fannush Shofi Akbar, S.ST.}
\var{\nikSatu}{20910026}
\var{\pembimbingDua}{Risdilah Mimma Untsa, S.ST., M.T.}
\var{\nikDua}{20910025}
% 
% Alias untuk memudahkan alur penulisan paa saat menulis laporan
\var{\saya}{Penulis}

%-----------------------------------------------------------------------------%
% Judul Setiap Bab
%-----------------------------------------------------------------------------%
% 
% Berikut ada judul-judul setiap bab. 
% Silahkan diubah sesuai dengan kebutuhan. 
% 
\Var{\kataPengantar}{Kata Pengantar}
\Var{\babSatu}{Pendahuluan}
\Var{\babDua}{Landasan Teori}
\Var{\babTiga}{Metodologi Penelitian}
\Var{\babEmpat}{Pengumpulan dan Pengolahan Data}
\Var{\babLima}{Analisis dan Pembahasan}
\Var{\babEnam}{Kesimpulan dan Saran}


% Awal bagian penulisan laporan
\begin{document}

% Sampul Laporan

\begin{titlepage}
    \begin{center}      
        % judul thesis harus dalam 14pt Times New Roman
        \bo{\Judul} \\[0.55cm]
        
        \vspace*{1 cm}
        \textit{\bo{\JudulInggris}} \\[1.5cm]    
        % harus dalam 14pt Times New Roman
        %\bo{\Type}
        \textbf{TUGAS AKHIR}\\
        Diajukan untuk memenuhi salah satu syarat memperoleh gelar sarjana\\
        Dari Program Studi S1 Teknik Telekomunikasi \\
        Direktorat Kampus Surabaya \\
        Universitas Telkom

        \vspace*{1 cm}       
        % penulis dan npm
        Disusun oleh:\\
        \bo{\Penulis} \\
        \bo{\nim} \\

        \vspace*{1.0cm}
        
        \begin{figure}
            \begin{center}
                \includegraphics[scale=1]{pics/pengantar/TelU.png}
            \end{center}
        \end{figure}
        \vspace*{1.0cm}
        % informasi mengenai fakultas dan program studi
        \bo{
            PROGRAM STUDI SARJANA TEKNIK TELEKOMUNIKASI\\
            DIREKTORAT KAMPUS SURABAYA\\
        	UNIVERSITAS TELKOM\\
        	\Kampus\\
        	\Tahun
        }
    \end{center}
\end{titlepage}


\pagenumbering{gobble}

% Gunakan penomoran halaman romawi
\pagenumbering{roman}

% setelah bagian ini, halaman dihitung sebagai halaman ke 3
\setcounter{page}{2}

% Halaman pengesahan
\addChapter{LEMBAR PENGESAHAN}
\chapter*{}

    \begin{center}
    \textbf{LEMBAR PENGESAHAN}\\
	\vspace*{0.15 cm}
    \textbf{TUGAS AKHIR}\\

	\vspace*{0.15 cm}
	
	\textbf{\Judul}

	\vspace*{0.5 cm}
	
	Telah disetujui dan disahkan untuk memenuhi sebagian syarat memperoleh gelar\\ 
	Sarjana pada Program Studi Teknik Telekomunikasi \\
	Direktorat Kampus Surabaya\\
	Universitas Telkom

	\vspace*{0.5cm}
	\textbf{Disusun oleh:}\\
	\textbf{\Penulis} \\
	\textbf{\nim} \\
	
	\vspace*{0.5 cm}
		\textbf{Surabaya, 05 Februari 2025}\\
		\textbf{Menyetujui,}
	\end{center}

	\begin{tabular}{c l c c c c r}
		1. & & & \\
		& \textbf{\underline{Dr. Fannush Shofi Akbar, S.ST.}}& & & & & (Pembimbing I) \\
		& NIP. 20910026 & & & \\
		2.  & & & \\
		& \textbf{\underline{Risdilah Mimma Untsa, S.ST., M.T.}}& & & & &(Pembimbing II) \\
		& NIP. 20910025& & & \\
		3.  & & & \\
		& \textbf{\underline{Nilla Rachmaningrum, S.T., M.T.}}& & & & &(Penguji I) \\
		& NIP. 17780080& & & \\
		4.  & & & \\
		& \textbf{\underline{Muhsin, S.T., M.T., Ph.D.}}& & & & &(Penguji II) \\
		& NIP. 19940001& & & \\
	\end{tabular}
 
    \null\hfill \textbf{Kaprodi Teknik Telekomunikasi,}\\
	\\ \\
	\null\hspace{8.1cm}\textbf{\underline{Dr. Fannush Shofi Akbar, S.ST.}}\\
	\null\hspace{8.1cm}NIP. 20910026

% Halaman pernyataan orisinalitas
\addChapter{LEMBAR PERNYATAAN ORISINALITAS}
\chapter*{}

    \begin{center}
    \textbf{LEMBAR PERNYATAAN ORISINALITAS}\\
    \end{center}
    
    \begin{tabular}{ll}
    \multirow{6}{*}{\includegraphics[scale=0.1]{pics/pengantar/BimaFoto.png}}
    Nama & :\hspace*{0.2 cm}\penulis \\
    \hspace{3.25cm}NIM & :\hspace*{0.2 cm}\nim \\
    \hspace{3.25cm}Alamat & :\hspace*{0.2 cm}\alamat \\
    \hspace{3.25cm}No. Telepon & :\hspace*{0.2 cm}\tlp \\
    \hspace{3.25cm}Email & :\hspace*{0.2 cm}\email \\
    \end{tabular}
    
    \vspace*{2 cm}
    Menyatakan bahwa Tugas Akhir ini merupakan karya orisinal saya sendiri. Atas pernyataan ini, saya siap menanggung risiko atau sanksi yang dijatuhkan kepada saya apabila kemudian ditemukan adanya pelanggaran terhadap kejujuran akademik atau etika keilmuan dalam karya ini, atau ditemukan bukti yang menunjukkan ketidakaslian karya ini. 
    
    \vspace*{1 cm}
    
    \begin{tabular}{cl}
    & \kampus, \tanggalPengesahan \\
    & \\
    & \\
    & \penulis \\
    \cline{2-2}
    &  \nim\\
    \end{tabular}

% Abstrak Bahasa Indonesia dan Bahasa Iggris
\addChapter{ABSTRAK}

\chapter*{Abstrak}
\vspace*{0.7cm}

Teknologi penginderaan semakin diperlukan di masa mendatang, salah satu teknik penginderaan dengan berbagai keunggulan adalah dengan radar. Radar \textit{Frequency Modulated Continuous Wave} yang populer digunakan merupakan salah satu teknik yang ramai digunakan. Implementasi teknik ini kerap ditemukan di berbagai bidang, mulai dari otomotif hingga kesehatan.\\

Pada penelitian ini, telah dirancang radar FMCW berbasis \textit{Software Defined Radio} dengan menggunakan GNURadio untuk melakukan deteksi, estimasi jarak, dan kecepatan dari suatu objek. Spesifikasi sistem radar yang dirancang ada pada frekuensi pembawa 3.1 GHz, dengan bentuk modulasi \textit{Triangular}. Direncanakan bahwa implementasi akan dilakukan dengan dua unit USRP menggunakan antena \textit{log periodic}.



\vspace*{0.2cm}

\noindent \textbf{Kata Kunci}: \textit{FMCW}, \textit{Radar}, \textit{GNURadio}\\ 

\newpage

\chapter*{ABSTRACT}
\vspace*{0.7cm}

\textit{Sensing Technology will become an important part of the future, one of its purposes is for object detection, velocity estimation, and ranging for speed trap camera on the side of the road to detect the velocity of vehicle so they wont surpass the chosen speed limit. Radar technology is one of sensing technology. Frequency Modulated Continuous Wave Radar is a popular technique that many people use. Many implementation of this technique is often use in various field, from automotive to health.}

\textit{In this research, a Software Defined Radio based FMCW radar has been design with GNURadio for detection, range estimation, and velocity of an object. The specification of system radar that has been designed is on 5.8 GHz carrier frequency, with a sawtooth modulation. The implementation use one unit of USRP with log periodic antenna. The result for range testing show a good quality from 6 meter above. The prediction accuracy for 6 meter is 82.86\% and the accuracy prediction for 9 meter is around 93.56\%. For range estimation at 3 meter, the prediction accuracy is at its worst, at about -402.72\%. The velocity experiment does not show a good quality}.

\noindent \textbf{Keywords}: \textit{FMCW}, \textit{Radar}, \textit{GNURadio}, \textit{USRP}, \textit{Detection}, \textit{Estimation}\\ 
\newpage

% Kata Pengantar
\addChapter{\kataPengantar}
%-----------------------------------------------------------------------------%
\chapter*{\kataPengantar}
%-----------------------------------------------------------------------------%
Kata Pengantar

\todo{tulis kata pengantar di sini}
 
\vspace*{0.1cm}
\begin{flushright}
Surabaya, \tanggalPengesahan\\[0.1cm]
\vspace*{1cm}
\penulis

\end{flushright}

%\addChapter{UCAPAN TERIMA KASIH}
%\include{ucapan}

% Daftar isi, gambar, dan tabel
\tableofcontents
\clearpage
\listoffigures
\clearpage
\listoftables
\clearpage
% Daftar Rumus
%\addChapter{DAFTAR RUMUS}
%\include{daftarRumus}
% Daftar Lampiran
\addChapter{DAFTAR LAMPIRAN}
%-----------------------------------------------------------------------------%
\chapter*{Daftar Lampiran}
%-----------------------------------------------------------------------------%
\noindent
Lampiran A: Pengujian Pola Radiasi Antena\\
Lampiran B: Kode Program Pengolahan Data
\addChapter{DAFTAR ISTILAH}
%-----------------------------------------------------------------------------%
\chapter*{Daftar Istilah}
%-----------------------------------------------------------------------------%

\begin{tabular}{l c c c c l}
    Radar & & & & = & \textit{Radio Detection and Ranging} \\
    FMCW & & & & = & \textit{Frequency Modulated Continuous Wave} \\
    USRP & & & & = & \textit{Universal Software Radio Peripheral} \\
    SDR & & & & = & \textit{Software Defined Radio} \\
    LFM & & & & = & \textit{Linear Frequency Modulated} \\
    FFT & & & & = & \textit{Fast Fourier Transform} \\
    RMSE & & & & = & \textit{Root Mean Square Error} \\
    
\end{tabular}

% Gunakan penomeran Arab (1, 2, 3, ...) setelah bagian ini.
\pagenumbering{arabic}
% Bab 1 : Pendahuluan
%-----------------------------------------------------------------------------%
\chapter{\babSatu}
%-----------------------------------------------------------------------------%
%-----------------------------------------------------------------------------%
\section{Latar Belakang}
%-----------------------------------------------------------------------------%
Untuk melakukan pendeteksian objek, banyak cara yang dapat dilakukan agar hal itu bisa dicapai. Seperti contohnya adalah dengan menggunakan pengolahan visual dari hasil tangkapan kamera untuk melakukan analisis video, apalagi dengan menggunakan \textit{multi-camera network} \cite{Zhang2015}. Adapula penggunaan gelombang suara yang memanfaatkan frekuensi suara pada jarak ultrasonik untuk mendeteksi objek dan jarak dengan menggunakan mikrokontroler dan sensor ultrasonik \cite{Biswas2020}. Teknik lain yang menjadi alternatif adalah penggunaan gelombang elektromagnetik untuk mendeteksi objek dan jarak suatu benda dengan menggunakan radar. 

Radar sendiri adalah singkatan dari \textit{radio detection and ranging} yang berarti bahwa fokus kegunaan radar adalah pada pendeteksian dan estimasi jarak suatu benda. Dibandingkan dengan teknik pengukuran lain, keunggulan dari penggunaan radar adalah mampu mendeteksi objek pada jarak yang jauh serta dapat menembus kabut. Keunggulannya tersebut adalah alasan awal digunakannya radar dahulu kala, yaitu  pada medan perang untuk mendeteksi pasukan sebelum nampak sehingga dapat melakukan persiapan terlebih dahulu. 

\begin{figure}
	\begin{center}
		\includegraphics[scale=0.5]{pics/bab1/AplikasiRadar.png} 
		\caption[Penggunaan Radar Otomotif]{Penggunaan Radar Otomotif}
		\label{pic:aplikasiRadarKini}
	\end{center}
\end{figure}

Namun, saat zaman semakin modern dan peperangan mulai berkurang, maka radar pun beralih fungsi. Seperti contohnya radar pendeteksi cuaca yang digunakan oleh badan klimatologi untuk memudahkan prediksi cuaca, radar pada menara pengawas bandara yang berguna dalam memonitor pergerakan pesawat di udara, dan radar pendeteksi objek pada kendaraan otomotif yang berguna untuk mendeteksi objek dan mencegah tabrakan seperti pada gambar \ref{pic:aplikasiRadarKini}.

Karena kemampuan radar dalam melakukan deteksi dan estimasi jarak tersebut, maka riset untuk mengembangkan implementasi radar dengan berbagai teknik semakin banyak \cite{Jia2020,Xia2021,MoraHuaman2020,Sundaresan2015}. Salah satu diantaranya adalah implementasi \textit{Real-Time Frequency Modulated Continous Wave Radar} yang dikembangkan dengan GNURadio dan digunakan pada \textit{Software Defined Radio} \cite{Sundaresan2015}. Teknik \textit{Frequency Modulated Continous Wave} atau yang disingkat dengan FMCW merupakan teknik transmisi secara kontinyu dari radar yang dapat memiliki energi yang lebih tinggi dengan \textit{peak power} yang lebih rendah \cite{Stasiak2017}. FMCW sangat populer digunakan pada industri, seperti untuk mendeteksi objek bawah tanah \cite{Macasero2018}, pada sistem pengawasan maritim \cite{Lestari2017}, dan bidang otomotif  karena dapat bertahan pada berbagai cuaca, dapat menghasilkan performa yang sangat baik, dapat memprediksi jarak dan kecepatan suatu objek \cite{Deng2017}. 

Sedangkan \textit{Software Defined Radio}, atau dalam kasus ini Radar, merupakan penggunaan fungsionalitas dari sistem radar yang diatur lewat \textit{Software} dengan maksud untuk memvirtualisasikan \textit{hardware} dan membuat manajemen pemrograman yang dilakukan menjadi lebih mudah \cite{Zeng2019}. Dengan menggunakan SDR lewat \textit{Universal Software Radio Peripheral} (USRP) sebagai perangkat kerasnya, maka proses riset dan pengembangan menjadi lebih murah, dikarenakan tidak diperlukannya fabrikasi material tiap uji coba pada frekuensi tertentu. Peneliti hanya perlu memprogram USRP yang dimilikinya untuk menghasilkan frekuensi tertentu yang mereka inginkan. Salah satu alat yang dapat digunakan dalam melakukan pemrograman terhadap USRP adalah GNURadio.


GNURadio merupakan aplikasi tak berbayar yang berada dibawah lisensi \textit{GNU General Public License} untuk mempelajari pembuatan dan pengimplementasian sistem \textit{software defined radio}. Dengan melakukan pemrograman pada GNURadio untuk melakukan antarmuka dengan USRP yang dimiliki, peneliti dapat menentukan berapa frekuensi hingga \textit{sampling rate} yang diinginkan \cite{Prabaswara2011}.

Oleh karena itu, pada proposal ini dilakukan “\judul” sehingga dapat membuktikan bahwa sistem yang dirancang dapat melakukan pendeteksian objek dan estimasi jarak.

%-----------------------------------------------------------------------------%
\section{Rumusan Masalah}
%-----------------------------------------------------------------------------%
Dari latar belakang yang telah dipaparkan diatas, maka ditemukannya rumusan masalah, yaitu:
\begin{enumerate}
	\item Bagaimana rancangan sistem radar FMCW berbasis USRP B210 menggunakan GNURadio? 
	\item Bagaimana sistem radar FMCW berbasis USRP B210 yang telah dirancang dapat mendeteksi, mengestimasi jarak, dan kecepatan objek?
	\item Bagaimana tingkat keakurasian dari sistem radar FMCW pada USRP dalam mendeteksi objek, melakukan estimasi jarak, dan kecepatan?
\end{enumerate} 

%-----------------------------------------------------------------------------%
\section{Tujuan dan Manfaat}
%-----------------------------------------------------------------------------%
Dari rumusan masalah yang sudah didapatkan, maka bisa diambil beberapa tujuan yang ingin dicapai oleh penulis, yaitu:

\begin{enumerate}
	\item Untuk melakukan perancangan sistem radar FMCW berbasis USRP B210 menggunakan GNURadio.
	\item Untuk melakukan pengujian deteksi, estimasi jarak, dan kecepatan objek dari sistem radar FMCW pada USRP B210.
	\item Untuk mengetahui tingkat keakurasian pendeteksi objek, estimasi jarak, dan kecepatan menggunakan radar FMCW pada USRP.
\end{enumerate}

%-----------------------------------------------------------------------------%
\section{Batasan Permasalahan}
%-----------------------------------------------------------------------------%
Hal yang akan dilakukan dalam penelitian ini adalah.
\begin{enumerate}
	\item Parameter yang diidentifikasi pada rancang bangun ini adalah resolusi jarak, tingkat keakurasian, dan kecepatan.
	\item Pengujian sistem dengan menggunakan USRP B210 untuk melakukan pendeteksian objek dan estimasi jarak.
	\item Perangkat lunak yang digunakan adalah GNURadio.
	\item Antena yang digunakan adalah antena \textit{Log Periodik}
	\item Frekuensi kerja radar pada 3 GHz.
	\item Objek deteksi adalah kendaraan empat roda.
\end{enumerate}

%-----------------------------------------------------------------------------%
\section{Manfaat}
%-----------------------------------------------------------------------------%
Manfaat yang diharapkan dari hasil penelitian terkait dengan penelitian ini adalah. 
\begin{enumerate}
	\item Menguji keakurasian dari sistem FMCW Radar lewat estimasi jarak dan deteksi objek.
	\item Menjadi referensi dalam implementasi FMCW Radar pada berbagai macam industri.
\end{enumerate}

%-----------------------------------------------------------------------------%
\section{Metode Penelitian}
%-----------------------------------------------------------------------------%
Dalam melakukan pengerjaan Tugas Akhir yang diajukan, penyelesaian yang digunakan adalah dengan beberapa pendekatan yaitu: studi literatur, simulasi, analisis statistik, perancangan, dan implementasi.

%-----------------------------------------------------------------------------%
\section{Jadwal Penelitian}
%-----------------------------------------------------------------------------%
Untuk memastikan proposal ini berjalan dengan lancar, maka diperlukannya penentuan capaian yang ingin diraih pada suatu periode yang sudah ditentukan. Dengan teraihnya capaian tersebut maka tahapan selanjutnya dapat mulai dilakukan.

\begin{center}
	\begin{longtable}{|c|m{3.8cm}|c | m{3cm} |m{3cm}|}
		\caption{Agenda Penelitian}
		\label{tab:Agenda}\\
		\hline
		No. & Deskripsi Tahapan & Durasi & Tanggal & \textit{Milestone} \\
		\hline
		1. & Desain Sistem & 1 bulan & 1 September 2024 - 30 September 2024 & Diagram blok dan simulasi \\
		\hline
		2. & Implementasi dan pengujian& 1 bulan & 1 Oktober 2024 - 31 Oktober 2024 & Pengujian sistem selesai \\
		\hline
		3. & Penyusunan laporan Tugas Akhir & 2 minggu & 1 November 2024 - 15 November 2024 & Buku Tugas Akhir selesai \\
		\hline
	\end{longtable}
\end{center}


% Bab 2 : Landasan Teori
%-----------------------------------------------------------------------------%
\chapter{\babDua}
%-----------------------------------------------------------------------------%

%-----------------------------------------------------------------------------%
\section{Kajian Penelitian Terkait}
%-----------------------------------------------------------------------------%
Banyak sekali referensi yang menjadi bagian besar dalam tertulisnya proposal ini, referensi tersebut terdiri atas berbagai macam jenis literatur dari sumber yang dapat diakses secara daring. Tak sedikit pula literatur tersebut menjadi alasan besar latar belakang dari proposal ini dilahirkan, berikut adalah beberapa penelitian terdahulu yang menjadi referensi dalam melakukan penyusunan proposal ini :

% Penelitian Terdahulu
\begin{center}
	\newcolumntype{L}[1]{>{\raggedright\let\newline\\\arraybackslash\hspace{1pt}}m{#1}}
	\newcolumntype{C}[1]{>{\centering\let\newline\\\arraybackslash\hspace{1pt}}m{#1}}
	\newcolumntype{R}[1]{>{\raggedleft\let\newline\\\arraybackslash\hspace{1pt}}m{#1}}
	
	\begin{longtable}{|>{\centering\arraybackslash}p{0.6cm}|p{4cm}|p{4cm}|p{4cm}|}
	\caption{Penelitian Terdahulu}
	\label{tab:PenelitianDulu}\\
	
	\hline
	\textbf{No.} & \textbf{Judul} & \textbf{Penulis} & \textbf{Hasil}\\
	\hline
	
	1.& \textit{Multitarget Physical Activities Monitoring and Classification Using a V-Band FMCW Radar} (2023)
	& Rizzi Varela, Victor G.; Rodrigues, Davi V. Q.; Zeng, Leya; Li, Changzhi
	& - \\ \hline
	
	2. & \textit{A Short-Range FMCW Radar-Based Approach for Multi-Target Human-Vehicle Detection} (2022)
	& Tavanti, Emanuele; Rizik, Ali; Fedeli, Alessandro; Caviglia, Daniele D.; Randazzo, Andrea 
	& - \\ \hline
	
	3. & \textit{FMCW Radar With Enhanced Resolution and Processing Time by Beam Switching} (2021)
	& Hilario Re, Pascual D.; Comite, Davide; Podilchak, Symon K.; Alistarh, Cristian A.; Goussetis, George; Sellathurai, Mathini; Thompson, John; Lee, Jaesup
	& - \\ \hline
	
	4. & \textit{An X–band FMCW Radar Demonstrator Based on an SDR Platform} (2020)
	& Dabrowski, Grzegorz; Stasiak, Krzysztof; Drozdowicz, Jedrzej; Gromek, Damian; Samczynski, Piotr
	& - \\ \hline
	
	5. & \textit{Single Target Recognition Using a Low-Cost FMCW Radar Based on Spectrum Analysis} (2020)
	& Rizik, Ali; Tavanti, Emanuele; Vio, Roberto; Delucchi, Alessandro; Chible, Hussien; Randazzo, Andrea; Caviglia, Daniele D.
	& - \\ \hline
	
	6. & \textit{Educational Low-Cost C-Band FMCW Radar System Comprising Commercial Off-the-Shelf Components for Indoor Through-Wall Object Detection} (2021)
	& Jeong, Hyunmin; Kim, Sangkil
	& - \\ \hline
	
	7. & \textit{Modified FMCW system for non-contact sensing of human respiration} (2020)
	& Pramudita, Aloysius Adya; Suratman, Fiky Y.; Arseno, Dharu
	& - \\ \hline
	
	8. & \textit{FMCW Radar for Noncontact Bridge Structure Displacement Estimation} (2023)
	& Pramudita, Aloysius Adya; Lin, Ding-Bing; Dhiyani, Azizka Ayu; Ryanu, Harfan Hian; Adiprabowo, Tjahjo; Yudha, Erfansyah Ali
	& - \\ \hline
	
	9. & \textit{A Novel Scheme of High-Precision Heart Rate Detection With a mm-Wave FMCW Radar} (2023)
	& Zhou, Min; Liu, Yunxue; Wu, Shie; Wang, Chengyou; Chen, Zekun; Li, Hongfei
	& - \\ \hline
	\end{longtable}
\end{center}
%-----------------------------------------------------------------------------%
%\section{Teori Dasar}
%-----------------------------------------------------------------------------%

\section{Radar}

Penggunaan gelombang elektromagnetik sebagai sarana untuk mendeteksi objek adalah konsep dasar dari radar. Radar sendiri merupakan singkatan dari \textit{Radio Detection and Ranging}, dari situ sangat nampak sekali tujuan dari penggunaan alat ini, yaitu untuk mendeteksi sesuatu dan mengukur jarak dengan menggunakan gelombang radio. 

Cara kerja dari radar adalah dengan memancarkan gelombang di dalam ruang bebas yang kemudian radar akan mendeteksi gelombang pantulan dari objek tersebut. Adanya gelombang yang terpantul ini tidak hanya menunjukkan keberadaan dari suatu objek, namun dengan membandingkan gelombang pantulan yang diterima dengan gelombang yang dikirimkan maka informasi tentang objek yang terdeteksi dapat didapat \cite{Skolnik2001}.

\begin{equation}
	R = \frac{cT_{R}}{2}
	\label{eq:PersRadar}
\end{equation}

Persamaan \ref{eq:PersRadar} menjelaskan jarak antara target dengan antena, dengan $T_{R}$ sebagai waktu sinyal radar bergerak secara bolak balik dari dan menuju objek. Karena radar memakai gelombang elektromagnetik, maka $c$ memiliki kecepatan yang sama dengan cahaya, yaitu $3 \times 10 ^{8}$.

Pada gambar \ref{pic:skemaRadar} berikut, skema dan konsep dasar dari cara kerja radar dapat diamati. Terlihat bahwa sinyal yang dikirimkan akan mengenai target, dalam kasus ini adalah pesawat, lalu sinyal yang mengenai objek akan kembali dengan sinyal yang lebih kecil dengan amplitudo yang lebih rendah. Perubahan pada gelombang yang terpantul dapat menggambarkan perilaku yang sedang ditunjukkan oleh objek yang di deteksi, mulai dari pengurangan amplitudo hingga pergeseran fasa.

\begin{figure}
	\begin{center}
		\includegraphics[scale=0.5]{pics/bab2/skemaradar.png} 
		\caption[Skema Dasar Radar]{{Skema Dasar Radar} \cite{Skolnik2001}}
		\label{pic:skemaRadar}
	\end{center}
\end{figure}

Gambar \ref{pic:blokdiagram} menunjukkan blok diagram dari sistem radar pulsa sederhana. Dapat dilihat beberapa komponen yang membentuk seluruh sistem radar, semua komponen ini memiliki perannya sendiri sehingga proses pengiriman dan pendeteksian sinyal dapat dilakukan.  Bila seluruh sistem bekerja dengan baik, maka proses yang ditunjukkan pada penjelasan skema dasar radar dapat berjalan dengan lancar.

\begin{figure}
	\begin{center}
		\includegraphics[scale=0.35]{pics/bab2/blokdiagram.png} 
		\caption[Blok Diagram Radar]{{Blok Diagram Radar Sederhana \cite{Kingsley1999}}}
		\label{pic:blokdiagram}
	\end{center}
\end{figure}

Persamaan radar berguna untuk menghubungkan seluruh komponen yang terdapat pada suatu sistem radar. Hubungan di antara seluruh komponen tersebut akan di perlihatkan secara matematis, sehingga penerapannya pada suatu alat akan terlihat dengan jelas. Dengan adanya beberapa persamaan ini, proses desain suatu radar akan menjadi lebih mudah dilakukan dan prediksi dari hasil radar yang dirancang bisa didapatkan.

Salah satu persamaan pada radar adalah \textit{maximum unambiguous range}, yang bersimbol $R_{un}$, dengan $T_{p}$ sebagai periode pengulangan pulsa, yang bernilai $\frac{1}{f_{p}}$, dengan $f_{p}$ sebagai frekuensi pengulangan pulsa.

\begin{equation}
	R_{un} = \frac{cT_{p}}{2} = \frac{c}{2f_{p}}
\end{equation}

Bila antena yang digunakan dalam memancarkan gelombang elektromagnetika radar bersifat isotrop, maka kerapatan daya pada jarak $R$ dari radar akan sama dengan daya di transmisi ($P_{t}$) dibagi luas permukaan $4\pi R^{2}$ dari sebuah bola imajiner dengan radius $R$, atau dapat didefinisikan pula dengan.

\begin{equation}
	P = \frac{P_{t}}{4\pi R^{2}}
\end{equation}

Namun, pada kenyataannya radar seringkali menggunakan antena \textit{directive} untuk mengkonsentrasikan daya yang terradiasi pada arah tertentu. Maka kerapatan dayanya adalah

\begin{equation}
	\text{Kerapatan daya antena \textit{directional}} = \frac{P_{t} G}{4\pi R^{2}}
\end{equation}

Dengan G sebagai \textit{gain} maksimum suatu antena, yaitu

\begin{equation}
	G  = \frac{\text{Kerapatan daya maksimum dari antena \textit{directional}}}{\text{Kerapatan daya antena Isotrop \textit{lossless} dengan daya yang sama}}
\end{equation}

\textit{Radar Cross Section} atau yang sering disingkat dengan RCS merupakan daerah suatu objek dari target yang dapat terdeteksi oleh suatu radar. Area tersebut diperhitungkan dengan mempertimbangkan bentuk dari objek dan interaksinya dengan gelombang elektromagnetik. Pada \ref{pic:RCS} ditunjukkan beberapa sifat RCS dan persamaannya.

\begin{center}
	\begin{figure}[h!]
		\begin{subfigure}[b]{0.5\linewidth}
			\includegraphics[width=\linewidth]{pics/bab2/rcsBentuk.png}
			\caption{Bentuk dan Persamaan \textit{Radar Cross Section}}
		\end{subfigure}
		\begin{subfigure}[b]{0.5\linewidth}
			\includegraphics[width=\linewidth]{pics/bab2/rcsPola.png}
			\caption{Pola Radiasi dari \textit{Radar Cross Section}}
		\end{subfigure}
		\caption[\textit{Radar Cross Section}]{\textit{Radar Cross Section} \cite{ONeill2012}}
		\label{pic:RCS}
	\end{figure}
\end{center}

\section{Pengolahan Sinyal Radar}
Untuk mendapat suatu kesimpulan dari sinyal radar, maka dibutuhkan pengolahan sinyal radar yang tepat. Pengolahan sinyal tersebut dilakukan mulai dari pembentukan gelombang hingga pengambilan kesimpulan. 

\subsection{Bentuk Gelombang Radar}
\begin{figure}
	\begin{center}
		\includegraphics[scale=0.8]{pics/bab2/radarwaveform.png}
		\caption[Bentuk Gelombang Radar]{Bentuk Gelombang Radar \cite{Melvin2014}}
		\label{pic:bentukgelradar}
	\end{center}
\end{figure}
Bentuk gelombang radar dapat dibedakan menjadi dua kelas, yaitu radar dengan gelombang kontinyu dan radar pulsa. Seperti pada gambar \ref{pic:bentukgelradar}, kedua kelas tersebut masih dapat dibagi lagi kedalam beberapa teknik lain. Penggunaan salah satu jenis gelombang ditentukan berdasarkan kebutuhan radar yang akan di desain. 

Radar dengan gelombang pulsa akan memancarkan gelombang elektromagnetik dalam waktu singkat lalu jeda sejenak sesuai waktu yang ditentukan. Pada waktu jeda tersebut, radar akan mendeteksi sinyal pantul dari gelombang yang dikirim sebelumnya. Setelah waktu jeda berakhir, radar akan kembali memancarkan gelombang pulsa lagi. Radar dengan gelombang ini akan memancarkan gelombang elektromagnetik dengan \textit{power} yang tinggi. 

Sedangkan radar dengan gelombang kontinyu akan terus memancarkan serta menerima gelombang elektromagnetik tanpa henti dalam waktu yang bersamaan. Sehingga radar dengan gelombang kontinyu hanya digunakan pada sistem dengan \textit{power} yang rendah dengan jarak maksimum deteksi yang kecil. Hal ini disebabkan karena sering terjadinya kebocoran dari antena pengirim ke antena penerima. Alasan ini pula yang mendasari keputusan penggunaan \textit{power} yang rendah \cite{Scheer2015}.

\subsection{\textit{Frequency Modulated Continuous Wave Radar}}

Radar FMCW memancarkan sinyal yang bila terpantul objek, akan kembali terdeteksi. Hal ini dapat direalisasikan dengan blok diragram dari sistem radar FMCW seperti pada gambar \ref{pic:FMCWBlock}.  

\begin{figure}
	\begin{center}
		\includegraphics[scale=0.3]{pics/bab2/blokDiagramFMCW.png}
		\caption[Blok Diagram Radar FMCW]{Blok Diagram Radar FMCW}
		\label{pic:FMCWBlock}
	\end{center}
\end{figure}

Dari blok diagram tersebut, dapat dilihat bahwa sinyal yang diterima dicampurkan dengan sinyal yang dikirim, sehingga karena adanya \textit{delay} yang disebabkan oleh jarak gelombang bergerak, maka akan terdeteksi perbedaan frekuensi. Dengan begitu, perbedaan pada fasa dan frekuensi menjadi tolok ukur antara sinyal yang dikirim dengan sinyal yang di dapatkan kembali.

\begin{figure}
	\begin{center}
		\includegraphics[scale=0.3]{pics/bab2/txRxWave.jpg}
		\caption[FMCW Dalam Domain Waktu]{FMCW Dalam Domain Waktu}
		\label{pic:FMCWTime}
	\end{center}
\end{figure}

Oleh karena itu, salah satu karakteristik dari radar FMCW adalah bahwa jarak pengukuran dapat dihitung dengan membandingkan frekuensi sinyal yang diterima dengan sinyal yang ditransmisikan.   

\begin{equation} 
	R = \frac{c \Delta{t}}{2} = \frac{c \Delta{f}}{2(\frac{d(f)}{d(t)})}
	\label{eq:PersFMCW}
\end{equation}

Persamaan \ref{eq:PersFMCW} menunjukkan jarak (R) dengan objek yang terdeteksi. Yang mana $\Delta{t}$ adalah waktu tunda dalam detik, $\Delta{f}$ merupakan pergeseran frekuensi terukur dalam Hertz, dengan d(f)/d(t) sebagai pergeseran frekuensi dalam suatu periode. 

\begin{equation}
	R_{max} = \frac{F_{s} c}{2 K}
	\label{eq:MaxRange}
\end{equation}

Persamaan \ref{eq:MaxRange} menunjukkan jarak maksimum yang dapat di deteksi oleh radar FMCW. $F_{s}$ merupakan frekuensi \textit{sampling}, dan K adalah tingkat kenaikan frekuensi pada suatu periode yang dapat dihitung dengan persamaan \ref{eq:slope} yaitu mengurangi nilai maksimum frekuensi dengan nilai minimumnya, lalu membaginya dengan waktu \textit{sweep} (\textit{chirp}).

\begin{equation}
	K = \frac{f_{atas} - f_{bawah}}{T_{c}}
	\label{eq:slope}
\end{equation}

Selain itu, salah satu faktor penting yang perlu diperhitungkan dalam perancangan radar FMCW adalah resolusi jarak. Resolusi jarak sendiri merupakan kemampuan dari suatu radar dalam membedakan dua buah objek yang berdekatan.

\begin{equation}
	\Delta{R} = \frac{c}{2 BW}
	\label{eq:RangeRes}
\end{equation}

Persamaan \ref{eq:RangeRes} menjelaskan bahwa dengan membagi kecepatan cahaya dengan dua kali lebar pita frekuensi (\textit{Bandwidth}), maka resolusi jarak akan didapatkan.

\subsection{\textit{Linear Frequency Modulated Continuous Wave Radar}}
\textit{Linear Frequency Modulated}, yang juga sering disingkat sebagai LFM adalah teknik pengolahan sinyal yang dilakukan dengan menyapu frekuensi dari bawah ke atas (\textit{Up-Chirp}) atau dari atas ke bawah (\textit{Down-Chirp}). Dengan $f_{0}$ sebagai frekuensi tengah, dan dilakukan pada \textit{bandwidth} yang telah ditentukan. Teknik ini akan membantu pencapaian radar dengan resolusi yang lebih tinggi karena \textit{bandwidth} yang dicapai akan menjadi lebih tinggi.

Salah satu jenis gelombang LFM adalah \textit{Linear Triangular Frequency Modulation} yang ditunjukkan pada gambar \ref{pic:LFMTriangular}. Penggunaan jenis gelombang tersebut akan mempermudah proses evaluasi target.

\begin{figure}
	\begin{center}
		\includegraphics[scale=0.7]{pics/bab2/lfmTriangular.png}
		\caption[LFM Tipe Segitiga]{LFM Tipe Segitiga \cite{Jankiraman2018}}
		\label{pic:LFMTriangular}
	\end{center}
\end{figure}


Selain gelombang LFM segitiga, ada pula yang berbentuk seperti gigi gergaji (\textit{Sawtooth}) seperti gambar \ref{pic:lfmSaw}.

\begin{figure}
	\begin{center}
		\includegraphics[scale=0.65]{pics/bab2/lfmSawtooth.png}
		\caption[LFM Tipe Gigi Gergaji]{LFM Tipe Gigi Gergaji \cite{Jankiraman2018}}
		\label{pic:lfmSaw}
	\end{center}
\end{figure}

Seluruh teknik tersebut memiliki keunggulannya masing masing. Keunggulan tersebut didapat karena proses analisis yang berbeda. Pada LFM berbentuk gigi gergaji, maka hanya objek diam saja yang dapat dideteksi jarak dan kecepatannya seperti pada gambar \ref{pic:lfmDetail}. Namun bila menggunakan LFM berbentuk segitiga, maka objek yang bergerak dapat dideteksi jarak dan kecepatannya dalam waktu yang bersamaan.

\begin{figure}
	\begin{center}
		\includegraphics[scale=0.65]{pics/bab2/lfmDetail.png}
		\caption[Detail Analisis LFM \textit{Sawtooth}]{Detail Analisis LFM \textit{Sawtooth} \cite{Jankiraman2018}}
		\label{pic:lfmDetail}
	\end{center}
\end{figure}


\subsection{Teknik Pengolahan Sinyal}
Untuk melakukan pengambilan keputusan dari data yang diambil oleh radar, maka dibutuhkan langkah pengolahan yang benar dan mencakup berbagai hal. Beberapa parameter yang bisa diambil estimasinya adalah jarak dan kecepatan dari objek yang terdeteksi. Pada estimasi jarak, persamaan \ref{eq:RangeEst} dapat menjelaskan hubungan jarak dengan beberapa faktor yang mempengaruhinya.

\begin{equation}
	d_{0} = \frac{c f_{b}}{2 \mu} = \frac{c T_{c} f_{b}}{2 B}
	\label{eq:RangeEst}
\end{equation}

Pada persamaan tersebut, terdapat c sebagai kecepatan cahaya, $f_{b}$ adalah \textit{beat frequency} yang merupakan perbedaan pada frekuensi,  $\mu$ yang merupakan laju perubahan frekuensi pada suatu waktu (\textit{chirp rate}), dengan $T_{c}$ sebagai waktu \textit{Sweep}. Sedangkan untuk melakukan estimasi kecepatan terdapat pergeseran frekuensi akibat efek doppler, yang menjelaskan perubahan frekuensi suatu gelombang karena suatu objek sumber yang bergerak. Bila pergeseran doppler ($f_{d}$), dengan v sebagai kecepatan, dan $\lambda$ adalah panjang gelombang, maka didapatkan persamaan \ref{eq:velocity}.

\begin{equation}
	v = \frac{f_{d}}{2}\lambda
	\label{eq:velocity}
\end{equation}

\subsection{Perhitungan \textit{Error}}
Penghitungan galat dari radar yang telah didesain dapat dilakukan dengan menguji keakurasian dari hasil deteksi. Hasil akurasi deteksi radar dapat diuji dengan menggunakan \textit{Root Mean Square Error} (RMS E) dari \textit{Signal to Noise Ratio}, sesuai persamaan \ref{eq:sigmaRN}.

\begin{equation}
	\sigma_{RN} = \frac{RMS E}{\sqrt{2 SNR_{L}}}
	\label{eq:sigmaRN}
\end{equation}

Dengan nilai dari RMS E bisa didapat dengan persamaan \ref{eq:rmsE}.

\begin{equation}
	RMS E = \frac{\sqrt{\sum_{t = 1}^{k} (m(t)-n(t))^2}}{k}
	\label{eq:rmsE}
\end{equation}

Nilai dari k adalah jumlah data, dengan m sebagai hasil data berdasarkan simulasi, dan n adalah data sebenarnya. Dengan begitu, nilai akurasi deteksi radar dapat dihitung dengan persamaan \ref{eq:accRadar}.

\begin{equation}
	Akurasi = 1 - \sigma_{RN}
	\label{eq:accRadar}
\end{equation}

\section{\textit{Software Defined Radio}}
\textit{Software Defined Radio} atau yang sering disingkat menjadi SDR merupakan teknologi komunikasi berbasis nirkabel yang kegunaannya dapat ditentukan oleh perangkat lunak \cite{Anisah2018}. Sehingga dalam implementasinya, tidak perlu dilakukan perubahan perangkat keras baru bila ingin melakukan perubahan, baik dari segi standar, teknologi, dan layanan. Hanya dengan melakukan perubahan konfigurasi saja, lalu SDR akan langsung dapat digunakan. 

Dalam implementasinya, SDR membutuhkan \textit{Universal Software Radio Peripheral}, atau yang sering disingkat menjadi USRP merupakan \textit{hardware} yang merupakan bagian \textit{front end} pada arsitektur sistem SDR. USRP terdiri dari modul yang dapat terkoneksi dengan komputer sehingga memperbolehkan pemrograman dengan aplikasi seperti GNURadio dan LabVIEW \cite{Gulo2023}. 

Penggunaan USRP sangat memudahkan proses perancangan prototipe dan pengujian karena adanya antarmuka yang dapat mengkoneksikan USRP dengan antena dan berbagai macam bagian perangkat keras yang dibutuhkan.

\subsection{\textit{Universal Software Radio Peripheral}}

\textit{Universal Software Radio Peripheral} atau yang sering disingkat dengan USRP merupakan \textit{platform} yang digunakan dalam mengimplementasikan SDR. Di dalam USRP terdapat \textit{Field Programmable Gate Array} atau FPGA yang merupakan suatu \textit{Integrated Circuit} yang dapat diprogram. Pada hal ini, USRP adalah perangkat keras yang dapat menerima dan mentransmisikan gelombang radio.

Kemampuannya untuk berinteraksi dengan gelombang radio inilah, ditambah pula dengan kemudahannya untuk melakukan pemrograman terhadap USRP ini yang membuat alat ini terkenal di kalangan akademisi dan peneliti. Karena pelaksanaan dan pengembangan prototipe menjadi lebih mudah dengan menghapuskan keperluan pengadaan komponen dalam prototipe.

\begin{center}
	\begin{figure}[h!]
		\begin{subfigure}[b]{0.5\linewidth}
			\includegraphics[width=\linewidth]{pics/bab2/B210.jpg}
			\caption{USRP B210 dengan \textit{enclosure}}
		\end{subfigure}
		\begin{subfigure}[b]{0.5\linewidth}
			\includegraphics[width=7.3cm]{pics/bab2/B210Board.jpg}
			\caption{\textit{Board} USRP B210}
		\end{subfigure}
		\caption{USRP B210}
		\label{pic:gambarusrp}
	\end{figure}
\end{center}

Ada beberapa USRP yang ada di pasar, salah satu yang cukup seringkali digunakan adalah USRP buatan dari \textit{Ettus}. Salah satu serinya adalah B210 seperti pada gambar \ref{pic:gambarusrp}. Penggunaan seri ini tidak tanpa alasan, karena seperti yang dapat dilihat pada tabel \ref{tab:spekb210}, spesifikasi USRP ini cukup memenuhi kebutuhan riset pada frekuensi yang sering di gunakan, dengan kapabilitas pengolahan sampel yang baik.

\begin{longtable}{|c|c|c|c|}
	\caption{Spesifikasi \textit{USRP} B210}
	\label{tab:spekb210}\\
	\hline
	No. & Keterangan & Nilai & Satuan \\
	\hline
	1. & \textit{RF Coverage} & 70 - 6 & MHz - GHz \\
	\hline
	2. & \textit{Analog to Digital Converter Sample Rate} (maksimum) & 61.44 & MS/s \\
	\hline
	3. & \textit{Analog to Digital Resolution}  & 12 & bits	\\
	\hline
	4. &\textit{Analog to Digital Wideband SFDR} & 78 & dBc \\
	\hline
	5. & \textit{Digital to Analog Converter Sample Rate} (maksimum) & 61.44 & MS/s \\
	\hline
	6. & \textit{Digital to Analog Resolution}  & 12 & bits	\\
	\hline
	7. & \textit{Host Sample Rate} (16b) & 61.44 & MS/s \\
	\hline
	8. & \textit{Frequency Accuracy} & $\pm 2.0$ & ppm \\
	\hline
	9. &  \textit{W/ GPS Unlocked TCXO Reference} & $\pm 75$ & ppb \\
	\hline
	10. & \textit{W/ GPS Locked TCXO Reference} & $<$ 1 & ppb \\ 
	\hline
\end{longtable}

Dengan spesifikasi tersebut, maka USRP B210 memiliki kemampuan \textit{instantneous bandwidth} hingga 56 MHz pada transmisi 1 X 1 dan 30.72 MHz pada transmisi 2 X 2.

\subsection{\textit{GNURadio}}

\begin{figure}
	\begin{center}
		\includegraphics[scale=0.5]{pics/bab2/GNU.png} 
		\caption[Logo GNURadio]{Logo GNURadio}
		\label{pic:logoGnuRadio}
	\end{center}
\end{figure}
GNURadio adalah aplikasi yang dapat melakukan pemrograman terhadap USRP lewat antarmuka. GNURadio merupakan \textit{software open source} sehingga semua orang dapat mengakses, mengubah, dan membagikan \textit{source code} dari program tersebut secara bebas. Dengan menggunakan aplikasi ini, perubahan parameter pada USRP dapat dilakukan dengan mudah.

\begin{figure}
	\begin{center}
		\includegraphics[scale=0.35]{pics/bab2/blokDiagramGRC.png} 
		\caption[Contoh \textit{Flowgraph} GNURadio]{Contoh \textit{Flowgraph} GNURadio}
		\label{pic:contohBlokGRC}
	\end{center}
\end{figure}

Gambar \ref{pic:contohBlokGRC} adalah contoh blok diagram sistem (\textit{flowgraph}) yang sukses dibuat pada aplikasi GNURadio. Pada gambar \ref{pic:contohRunGRC} menunjukkan hasil bila desain sistem tersebut dijalankan.

\begin{figure}
	\begin{center}
		\includegraphics[scale=0.4]{pics/bab2/contohRunGRC.png} 
		\caption[Hasil Desain Sistem GNURadio]{Hasil Desain Sistem GNURadio}
		\label{pic:contohRunGRC}
	\end{center}
\end{figure}




% Bab 3 : Metodologi Penelitian
%-----------------------------------------------------------------------------%
\chapter{\babTiga}
%-----------------------------------------------------------------------------%

%-----------------------------------------------------------------------------%
\section{Alur Penelitian}
%-----------------------------------------------------------------------------%
Dalam suatu penelitian, terdapat urutan tahapan yang perlu dilakukan. Alur penelitian ini mengandung seluruh langkah yang harus ditempuh, mulai dari fase perancangan hingga tahap akhir penelitian.
 \begin{figure}
	\begin{center}
		\includegraphics[scale=0.65]{pics/bab3/FlowchartSempro.png} 
		\label{img:flowchart}
		\caption[\textit{Flowchart} Penelitian]{\textit{Flowchart} Penelitian}
	\end{center}
\end{figure}
Pada alur penelitian yang telah dirancang, terdapat beberapa tahap yang perlu dilakukan setelah penelitian dimulai dan sebelum penelitian diakhiri. Tiap tahapan yang telah dirancang harus dilaksanakan sebaik mungkin agar hasil yang diharapkan dapat tercapai.
	
\section{Penentuan Parameter}

Pada tahap ini parameter pengujian ditentukan sehingga hasil yang dicapai dapat dikatakan baik, sebagai berikut.

\begin{center}
	\begin{longtable}{| c | c | c |}
		\caption{Parameter Pengujian}
		\label{tab:paramUji}\\
		\hline
		No. & Parameter Pengujian		& Satuan\\ \hline
		1.  &Jarak	   					& m\\
		2.  &Kecepatan 					& m/s\\
		3.  &\textit{RMSE}				& -\\
		4.	&Nilai prediksi \textit{beat frequency}	& Hz \\
		5.	&Nilai prediksi \textit{doppler frequency shift} & Hz \\
		\hline
	\end{longtable}
\end{center}

\section{Perancangan Spesifikasi Sistem}
Pada tahap ini, dilakukan perancangan tentang penelitian yang diangkat, dalam konteks ini adalah radar. Sehingga perlu dilakukannya penentuan spesifikasi radar berdasarkan perangkat keras yang digunakan. Penelitian ini menggunakan alat USRP berseri B210.  Spesifikasi dari alat ini akan dijelaskan pada tabel berikut.

\begin{center}
	\begin{longtable}{| c | c | c |}
		\caption{Spesifikasi Sistem Radar}
		\label{tab:spekRadar}\\
		\hline
		No. & Spesifikasi 					& Keterangan\\\hline
		1.  & USRP 							& B210\\
		2.  & \textit{Center Frequency}  	& 3000 MHz \\
		3.  & \textit{Bandwidth} 			& 50 MHz \\
		4.	& Bentuk Modulasi				& \textit{Triangular}\\
		5.  & Jarak Maksimum 				& 150 km \\
		6.  & Resolusi Jarak 				& 3 m \\
		7.  & Kecepatan Maksimum			& 15 $m/s$ \\
		8.  & Resolusi Kecepatan 			& 1 $m/s$\\
		9.	& Durasi \textit{Chirp}			& 0.001667 s\\
		10.	& \textit{Chirp Rate}			& 30000 MHz\\
		\hline
	\end{longtable}
\end{center}

\begin{itemize}
	\item Hitung panjang gelombang ($\lambda$) dari frekuensi tengah yang sudah ditentukan yaitu 3 GHz.
	\begin{align*}
		\lambda &= \frac{c}{F_{c}}\\
		\lambda &= \frac{3 \cdot 10^{8}}{3 \cdot 10^{9}}\\
		\lambda &= 0.1 m
	\end{align*}

	\item Menghitung resolusi jarak berdasarkan persamaan \ref{eq:RangeRes} dan dengan menentukan \textit{bandwidth} bernilai 50 MHz, maka.
		\begin{align*}
			R_{res} &= \frac{c}{2 BW} \\
			R_{res} &= \frac{3 \cdot 10^{8}}{2 \cdot 50 MHz}\\
			R_{res} &= 3 m
		\end{align*}
	\item Menghitung jarak maksimum yang dapat dideteksi oleh radar digunakanlah persamaan \ref{eq:MaxRange}, namun sebelumnya harus ditentukan terlebih dahulu nilai $\mu$, yang merupakan tingkat kenaikan frekuensi pada suatu periode sesuai dengan persamaan \ref{eq:chirpRate}, dengan nilai $T_{c}$ sesuai persamaan \ref{eq:chirpTime} dan nilai kecepatan maksimum ditentukan bernilai 15 m/s, maka.
	
	\begin{align*}
		T_{c} &= \frac{\lambda}{4 \cdot V_{max}}\\
		T_{c} &= \frac{0.1}{4 \cdot 15}\\
		T_{c} &= 0.001667
	\end{align*}

	\item 
	Sehingga nilai $\mu$ dapat dihitung menjadi.

		\begin{align*}
		\mu &= \frac{\textit{Bandwidth}}{T_{c}}\\
		\mu &= \frac{\textit{50 MHz}}{0.001667}\\
		\mu &= 30000 MHz/s
		\end{align*}

	\item 	
	Dengan jarak maksimum yang didapat adalah.
		\begin{align*}
		R_{max} &= \frac{F_{s} \cdot c}{2 \cdot \mu}\\
		R_{max} &= \frac{30 \cdot 10^{6} \cdot 3 \cdot 10^{8}}{2 \cdot 30000}\\
		R_{max} &= 150 m
		\end{align*}

	\item 
	Dengan $T_{f}$ sebagai durasi \textit{frame} bernilai 0.05 s maka resolusi kecepatannya.
		\begin{align*}
			V_{res} &= \frac{\lambda}{2 \cdot T_{f}}\\
			V_{res} &= \frac{0.1}{2 \cdot 0.05}\\
			V_{res} &= 1 m/s
		\end{align*}

\end{itemize}

\section{Implementasi Sistem}
Tahap implementasi ini dilakukan pada aplikasi GNURadio dan menghasilkan \textit{flow diagram} yang merepresentasikan langkah yang dilakukan pada USRP. \textit{Flow diagram} yang didesain sudah memenuhi spesifikasi sistem radar pada tabel \ref{tab:spekRadar}. 

Implementasi sistem akan dilaksanakan pada beberapa perangkat, mulai dari laptop, antena, dan USRP. Berikut detail perangkat yang akan digunakan pada saat implementasi guna mendapat hasil yang baik.

\begin{enumerate}
	\item \textit{IdeaPad Gaming 3 15ARH7} :
	\begin{figure}
		\begin{center}
			\includegraphics[scale=0.2]{pics/bab3/laptop.jpg} 
			\caption[Gambar Perangkat Laptop Yang Digunakan]{Gambar Perangkat Laptop Yang Digunakan}
			\label{pic:contohBlokGRC}
		\end{center}
	\end{figure}

	\begin{itemize}
		\item \textit{Processor} : AMD Ryzen 7 6800H dengan \textit{Radeon Graphics} 3.20 GHz
		\item \textit{Memory} : 8,00 GB (7,19 GB \textit{usable})
	\end{itemize}

	\item Perangkat \textit{Software Defined Radio} :
	\begin{figure}
		\begin{center}
			\includegraphics[scale=0.045]{pics/bab3/usrp2.jpg}
			\caption{Alat USRP B210}
			\label{img:logPeriodic}
		\end{center}
	\end{figure}
	\begin{itemize}
		\item Tipe : USRP B210 
		\item Jarak Frekuensi : 70 MHz - 6000 MHz 
	\end{itemize}

	\item Antena \textit{Log-periodic} :
	\begin{figure}
		\begin{center}
			\includegraphics[scale=0.05]{pics/bab3/logPeriodic.jpg}
			\caption{Antena \textit{Log Periodic} Pengujian}
			\label{img:usrpBoard}
		\end{center}
	\end{figure}
	\begin{itemize}
		\item Frekuensi : 800 MHz - 6000 MHz 
		\item Pola Radiasi : \textit{Directional}
		\item \textit{Gain} : 5.2 - 6.3 dB
	\end{itemize}
\end{enumerate}

	
\section{Pengambilan Data}
Pada tahap ini, pengambilan data dengan radar yang sudah didesain dan diimplementasikan pada USRP dilakukan. Pengujian dilakukan dengan menggunakan kendaraan roda empat sebagai objek yang akan dideteksi. Sehingga pengambilan data kecepatan dan prediksi jarak dapat dilakukan. Hasil prediksi jarak dan kecepatan radar akan dibandingkan dengan nilai aktual jarak pada kenyataan dan kecepatan tercatat pada \textit{speedometer}.

\begin{figure}
	\begin{center}
		\includegraphics[scale=0.55]{pics/bab3/skema.png}
		\caption{Skema Penelitian}
		\label{img:skema}
	\end{center}
\end{figure}

Data berupa nilai \textit{beat frequency} dan \textit{doppler frequency shift} yang sudah ditentukan sebagai parameter pengujian telah didapat dari hasil pengambilan data akan dibandingkan dengan nilai prediksi berdasarkan perhitungan. Dengan begitu, maka nilai RMSE dapat dihitung.

\begin{figure}
	\begin{center}
		\includegraphics[scale=0.35]{pics/bab3/petaPengujian.png}
		\caption{Lokasi Pengujian}
		\label{img:petaUji}
	\end{center}
\end{figure}

Pengambilan data akan dilaksanakan di lokasi lapangan Univertitas Telkom Surabaya yang beralamat Jl. Ketintang No.156, Ketintang, Kec. Gayungan, Surabaya, Jawa Timur 60231.

\section{Konfigurasi Pengujian}
Konfigurasi pengujian dilakukan sesuai dengan gambar \ref{img:skema}. Terdapat satu buah perangkat laptop yang terhubung dengan dua buah USRP, masing-masing USRP terhubung dengan antena \textit{Log-periodic}. USRP 1 berperan sebagai \textit{transmitter} sedangkan USRP 2 berperan sebagai \textit{receiver}.

\begin{figure}
	\begin{center}
		\includegraphics[scale=0.09]{pics/bab3/konfigurasiPengujian.jpg}
		\caption{Konfigurasi Pengujian}
		\label{img:konfigurasi}
	\end{center}
\end{figure}


% \section{Perhitungan Nilai Simulasi}
% \todo{Hitung Nilai $F_{b}$ dan $F_{d}$ dengan tabel}
% Dolore velit amet amet aliqua exercitation velit nulla ad eu. Ad voluptate minim fugiat et mollit commodo elit. Excepteur minim magna aute commodo consectetur velit aute et consectetur sit. Fugiat voluptate ea officia labore. Ut aute voluptate proident officia consequat nostrud aute nulla consequat enim. Lorem nostrud sit pariatur officia Lorem officia. Aliquip labore Lorem quis ea proident reprehenderit labore sit.
% Bab 4 : Pengumpulan dan pengolahan data
%-----------------------------------------------------------------------------%
\chapter{\babEmpat}
%-----------------------------------------------------------------------------%

%-----------------------------------------------------------------------------%
\section{Petunjuk Penggunaan}
%-----------------------------------------------------------------------------%


%%--------------------------------%
%\todo{
	%\section{Simulasi Sistem}
	%Masukkan gambar simulasi pada gnuradio\\ 
	%Masukkan gambar hasil yang dilakukan dengan simulasi (berupa time vs frequency).\\
	%Masukkan proses analisis yang dilakukan dengan hasil simulasi, seperti langkah konversi biner ke kompleks, penentuan data yang diambil,... pada matlab.\\
	%Lakukan desain blok dari hasil pada matlab menuju python.\\
	%Uji coba blok python yang telah dirancang pada simulasi\\
	%Bandingkan hasilnya dengan pada matlab
	%}
%%--------------------------------%

\begin{figure}
	\centering
	\includegraphics[width=0.3\textwidth]
		{pics/creative_common.png}
		\caption{CC-A-NC-SA 1.0 Generic}
	\label{fig:CC10}
\end{figure}

\textit{Template} tugas ini dipasangkan lisensi \textit{Creative Common---Attribution---Non Commercial---Share Alike 1.0 Generic}.
% Bab 5 : Analisis dan Pembahasan
%---------------------------------------------------------------
\chapter{\babLima}
%---------------------------------------------------------------

%---------------------------------------------------------------
\section{Hasil Pengolahan Data Jarak}
%---------------------------------------------------------------
Pada bagian ini, data dari \textit{beat frequency} akan disajikan dalam bentuk gambar. Data tersebut adalah hasil setelah diolah pada langkah pengolahan data. Setelah didapat nilai \textit{beat frequency}, selanjutnya dimasukkan kedalam fungsi \textit{beat2range} yang terdapat pada matlab.

\subsection{Data Jarak 3 Meter}

\begin{figure}
    \centering
    \begin{subfigure}[b]{0.45\textwidth}
        \centering
        \includegraphics[scale=0.35]{pics/bab5/Range/1_3.jpg}
        \caption{Nilai \textit{beat frequency} 3 meter pengulangan 1}
        \label{fig:pengambilan3_1}
    \end{subfigure}
    \hfill
    \begin{subfigure}[b]{0.45\textwidth}
        \centering
        \includegraphics[scale=0.35]{pics/bab5/Range/2_3.jpg}
        \caption{Nilai \textit{beat frequency} 3 meter pengulangan 2}
        \label{fig:pengambilan3_2}
    \end{subfigure}
    \vskip\baselineskip
    \begin{subfigure}[b]{0.6\textwidth}
        \centering
        \includegraphics[scale=0.35]{pics/bab5/Range/3_3.jpg}
        \caption{Nilai \textit{beat frequency} 3 meter pengulangan 3}
        \label{fig:pengambilan3_3}
    \end{subfigure}
    \caption{Hasil \textit{beat frequency} dari jarak 3 meter}
    \label{fig:pengambilan3}
\end{figure}

Pada gambar~\ref{fig:pengambilan3} telah dilampirkan tiga sampel hasil percobaan dari total sembilan pengambilan data jarak objek setelah diolah dengan matlab untuk mendapatkan nilai \textit{beat frequency} dari objek. Pada gambar~\ref{fig:pengambilan3_1} yang menunjukkan pengambilan data pengulangan pertama didapatkan hasil nilai \textit{beat frequency} pada frekuensi -514.062 Hz, pada gambar~\ref{fig:pengambilan3_2} yang menunjukkan pengambilan data pengulangan kedua didapatkan nilai -68.3594 Hz, dan pada pengambilan data pengulangan ketiga, nilai \textit{beat frequency} adalah -79.2969 Hz seperti pada gambar~\ref{fig:pengambilan3_3}.

Dengan menggunakan persamaan~\ref{eq:RangeEst}, konversi \textit{beat frequency} ke jarak dapat dilakukan, perhitungan konversi adalah sebagai berikut.

\begin{align*}
    d_{0} &= \frac{|-514.062| \cdot 3 \cdot 10^{8}}{2 \cdot (\frac{14 \cdot 10^{6}}{0.01})}\\
   d_{0} &= 55.08 \phantom{b} meter\\
   d_{1} &= \frac{|-68.3594| \cdot 3 \cdot 10^{8}}{2 \cdot (\frac{14 \cdot 10^{6}}{0.01})}\\
   d_{1} &= 7.32 \phantom{b} meter\\
   d_{2} &= \frac{|-79.2969| \cdot 3 \cdot 10^{8}}{2 \cdot (\frac{14 \cdot 10^{6}}{0.01})}\\
   d_{2} &= 8.5 \phantom{b} meter\\
\end{align*}

Penggunaan mutlak pada \textit{beat frequency} karena nilai jarak yang didapat adalah nilai absolut yang merepresentasikan jarak di depan radar, dengan penggunaan \textit{double sideband} pada gambar~\ref{fig:pengambilan3Meter} maka \textit{beat frequency} perlu di ubah agar tidak mendapatkan hasil negatif jarak.

Konversi dari nilai \textit{beat frequency} menuju jarak dilakukan dengan fungsi \textit{beat2range} pada matlab, didapat dari nilai \textit{beat frequency} gambar~\ref{fig:pengambilan3_1} bahwa nilai jarak yang didapat adalah 55.08 meter. Nilai jarak dari \textit{beat frequency} gambar~\ref{fig:pengambilan3_2} adalah 7.32 meter. Sedangkan pada gambar~\ref{fig:pengambilan3_3} nilai jarak ada pada 8.50 meter. Dapat dilihat bahwa pada ketiga pengujian, nilai prediksi jarak tidak linier dengan jarak asli. Hal ini menunjukkan kegagalan radar dalam memprediksi jarak objek pada 3 meter.

\subsection{Data Jarak 6 Meter}

\begin{figure}
    \centering
    \begin{subfigure}[b]{0.45\textwidth}
        \centering
		\includegraphics[scale=0.35]{pics/bab5/Range/1_6.jpg}
		\caption{Nilai \textit{beat frequency} 6 meter pengulangan 1}
		\label{fig:pengambilan6_1}
    \end{subfigure}
    \hfill
    \begin{subfigure}[b]{0.45\textwidth}
        \centering
		\includegraphics[scale=0.35]{pics/bab5/Range/2_6.jpg}
		\caption{Nilai \textit{beat frequency} 6 meter pengulangan 2}
		\label{fig:pengambilan6_2}
    \end{subfigure}
    \vskip\baselineskip
    \begin{subfigure}[b]{0.6\textwidth}
        \centering
		\includegraphics[scale=0.35]{pics/bab5/Range/3_6.jpg}
		\caption{Nilai \textit{beat frequency} 6 meter pengulangan 3}
		\label{fig:pengambilan6_3}
    \end{subfigure}
    \caption{Hasil \textit{beat frequency} dari jarak 6 meter}
    \label{fig:pengambilan6}
\end{figure}

Pada gambar~\ref{fig:pengambilan6} telah dilampirkan tiga dari sembilan sampel hasil percobaan pengambilan data jarak objek setelah diolah dengan matlab untuk mendapatkan nilai \textit{beat frequency} dari objek. Pada gambar~\ref{fig:pengambilan6_1} yang menunjukkan pengambilan data pengulangan pertama didapatkan hasil nilai \textit{beat frequency} pada frekuensi 65.625 Khz, pada gambar~\ref{fig:pengambilan6_2} yang menunjukkan pengambilan data pengulangan kedua didapatkan nilai 62.8906 kHz, dan pada pengambilan data pengulangan ketiga, nilai \textit{beat frequency} adalah 51.9531 kHz seperti pada gambar~\ref{fig:pengambilan6_3}.

Dengan menggunakan persamaan~\ref{eq:RangeEst} yang terkandung dalam fungsi yang tersedia pada matlab, maka didapat nilai jarak dari~\ref{fig:pengambilan6_1} adalah 7.03 meter. Pada hasil \textit{beat frequency} gambar~\ref{fig:pengambilan6_2}, konversi jarak yang didapat adalah 6.74 meter. Pada hasil pengujian ketiga seperti gambar~\ref{fig:pengambilan6_3}, hasil prediksi jarak adalah 5.57 meter. Dari hasil jarak yang sudah didapat, menunjukkan bahwa hasil prediksi radar bekerja dengan cukup baik. 

\subsection{Data Jarak 9 Meter}

\begin{figure}
    \centering
    \begin{subfigure}[b]{0.45\textwidth}
        \centering
		\includegraphics[scale=0.3]{pics/bab5/Range/1_9.jpg}
		\caption{Nilai \textit{beat frequency} 9 meter pengulangan 1}
		\label{fig:pengambilan9_1}
    \end{subfigure}
    \hfill
    \begin{subfigure}[b]{0.45\textwidth}
        \centering
		\includegraphics[scale=0.35]{pics/bab5/Range/2_9.jpg}
		\caption{Nilai \textit{beat frequency} 9 meter pengulangan 2}
		\label{fig:pengambilan9_2}
    \end{subfigure}
    \vskip\baselineskip
    \begin{subfigure}[b]{0.6\textwidth}
        \centering
		\includegraphics[scale=0.35]{pics/bab5/Range/3_9.jpg}
		\caption{Nilai \textit{beat frequency} 9 meter pengulangan 3}
		\label{fig:pengambilan9_3}
    \end{subfigure}
    \caption{Hasil \textit{beat frequency} dari jarak 9 meter}
    \label{fig:pengambilan9}
\end{figure}

Pada gambar~\ref{fig:pengambilan9} telah dilampirkan tiga sampel dari total sembilan hasil percobaan pengambilan data jarak objek setelah diolah dengan matlab untuk mendapatkan nilai \textit{beat frequency} dari objek. Pada gambar~\ref{fig:pengambilan9_1} yang menunjukkan pengambilan data pengulangan pertama didapatkan hasil nilai \textit{beat frequency} pada frekuensi 84.7656 Khz, pada gambar~\ref{fig:pengambilan9_2} yang menunjukkan pengambilan data pengulangan kedua didapatkan nilai 84.7656 kHz, dan pada pengambilan data pengulangan ketiga, nilai \textit{beat frequency} adalah 82.0312 kHz seperti pada gambar~\ref{fig:pengambilan9_3}.

Setelah nilai \textit{beat frequency} tiap pengulangan didapat, maka dilakukanlah konversi dengan fungsi \textit{beat2range}. didapat hasil jarak dari gambar~\ref{fig:pengambilan9_1} adalah 9.08 meter. Pada pengujian dari \textit{beat frequency} gambar~\ref{fig:pengambilan9_2}, nilai jarak juga didapat 9.08 meter. Pada pengujian ketiga dengan \textit{beat frequency} seperti gambar~\ref{fig:pengambilan9_3}, nilai jarak yang didapat adalah 8.79 meter. Hasil pengujian jarak 9 meter dari objek dengan menggunakan radar menunjukkan kemampuan radar yang dirancang sudah cukup baik dalam memprediksi jarak objek.
%---------------------------------------------------------------
\section{Hasil Pengolahan Data Kecepatan}
%---------------------------------------------------------------

Hasil prediksi kecepatan didapat dengan melakukan perhitungan dari nilai prediksi jarak yang terdeteksi dalam suatu waktu. Dengan mengamati frekuensi hasil \textit{conjugate}, akan didapat perubahan jarak objek yang didapat. 

\subsection{Data Kecepatan 5 km/h}

\begin{figure}
    \centering
    \begin{subfigure}[b]{0.45\textwidth}
        \centering
		\includegraphics[scale=0.4]{pics/bab5/Velocity/5MA.jpg}
		\caption{Objek menjauhi radar}
		\label{fig:pengambilan5MA}
    \end{subfigure}
    \hfill
    \begin{subfigure}[b]{0.45\textwidth}
        \centering
		\includegraphics[scale=0.4]{pics/bab5/Velocity/5MT.jpg}
		\caption{Objek mendekati radar}
		\label{fig:pengambilan5MT}
    \end{subfigure}
    \caption{Hasil Prediksi Kecepatan Objek 5 km/h}
    \label{fig:pengambilan5}
\end{figure}

Pada gambar~\ref{fig:pengambilan5MA} dan gambar~\ref{fig:pengambilan5MT}, data kecepatan dari objek telah di ambil dan ditampilkan. Nampak bahwa data kecepatan tidak didapatkan. Hal ini sesuai dengan resolusi kecepatan yang telah dilampirkan dalam tabel~\ref{tab:spekRadar} bahwa objek baru akan terdeteksi setelah kecepatan mencapai minimal 9.31 km/h. 

\subsection{Data Kecepatan 10 km/h}

\begin{figure}
    \centering
    \begin{subfigure}[b]{0.45\textwidth}
        \centering
		\includegraphics[scale=0.4]{pics/bab5/Velocity/10MA.jpg}
		\caption{Objek menjauhi radar}
		\label{fig:pengambilan10MA}
    \end{subfigure}
    \hfill
    \begin{subfigure}[b]{0.45\textwidth}
        \centering
		\includegraphics[scale=0.4]{pics/bab5/Velocity/10MT.jpg}
		\caption{Objek mendekati radar}
		\label{fig:pengambilan10MT}
    \end{subfigure}
    \caption{Hasil Prediksi Kecepatan Objek 10 km/h}
    \label{fig:pengambilan10}
\end{figure}

Data kecepatan dari objek bergerak telah di ambil dan ditampilkan pada gambar~\ref{fig:pengambilan10}. Grafik tersebut menunjukkan bahwa radar dapat melakukan hasil prediksi kecepatan dari objek. Dengan perbedaan paling mencolok dari kedua skema pengujian adalah saat objek menjauhi radar, maka nilai kecepatan ada pada negatif, sedangkan saat objek mendekati radar maka nilai kecepatan bernilai positif. Nilai prediksi kecepatan ada pada -9.8 km/h untuk objek menjauhi radar, sedangkan saat objek mendekati radar, nilai kecepatan terdeteksi adalah 10.3 km/h.

\subsection{Data Kecepatan 15 km/h}

\begin{figure}
    \centering
    \begin{subfigure}[b]{0.45\textwidth}
        \centering
		\includegraphics[scale=0.4]{pics/bab5/Velocity/15MA.jpg}
		\caption{Objek menjauhi radar}
		\label{fig:pengambilan15MA}
    \end{subfigure}
    \hfill
    \begin{subfigure}[b]{0.45\textwidth}
        \centering
		\includegraphics[scale=0.4]{pics/bab5/Velocity/15MT.jpg}
		\caption{Objek mendekati radar}
		\label{fig:pengambilan15MT}
    \end{subfigure}
    \caption{Hasil Prediksi Kecepatan Objek 15 km/h}
    \label{fig:pengambilan15}
\end{figure}

Gambar~\ref{fig:pengambilan15} menunjukkan hasil prediksi kecepatan objek 15 km/h. Dari kedua skema pengujian, hanya skema pengujian objek menjauhi radar yang terdeteksi. Pada prediksi kecepatan tersebut, nilai yang diprediksi adalah -18,03 km/h. Untuk objek yang mendekati radar, didapati bahwa kecepatan objek tidak terdeteksi. Sehingga dapat dinilai bahwa kemampuan radar untuk mendeteksi kecepatan objek pada kecepatan 15 km/h kurang baik.

\subsection{Data Kecepatan 20 km/h}

\begin{figure}
    \centering
    \begin{subfigure}[b]{0.45\textwidth}
        \centering
		\includegraphics[scale=0.4]{pics/bab5/Velocity/20MA.jpg}
		\caption{Objek menjauhi radar}
		\label{fig:pengambilan20MA}
    \end{subfigure}
    \hfill
    \begin{subfigure}[b]{0.45\textwidth}
        \centering
		\includegraphics[scale=0.4]{pics/bab5/Velocity/20MT.jpg}
		\caption{Objek mendekati radar}
		\label{fig:pengambilan20MT}
    \end{subfigure}
    \caption{Hasil Prediksi Kecepatan Objek 20 km/h}
    \label{fig:pengambilan20}
\end{figure}

Data dari kecepatan objek yang bergerak pada kecepatan 20 km/h telah dilampirkan pada gambar~\ref{fig:pengambilan20}. Dapat dilihat bahwa kemampuan radar dalam mendeteksi objek yang bergerak pada kecepatan 20 km/h tidak bekerja. Baik pada skema pengujian objek mendekati maupun menjauhi radar. Maka dari itu, dapat dinyatakan bahwa kemampuan radar yang telah didesain tidak dapat mendeteksi objek dengan kecepatan 20 km/h.

%---------------------------------------------------------------
\section{Analisis Hasil}
%---------------------------------------------------------------

Pada tabel~\ref{tab:estRange} telah dilampirkan data deteksi jarak dari pengumpulan dan pengolahan data yang telah dilakukan. Pada tabel tersebut, didapatkan nilai deviasi antara prediksi dan nilai asli yang kecil pada jarak 6 meter dan 9 meter. Sedangkan pada data jarak 3 meter, didapati bahwa nilai prediksi sangat buruk.

\begin{table}[H]
    \caption{Tabel hasil prediksi dan akurasi}
    \label{tab:estRange}
    \begin{tabular}{|c|ccccccccc|c|}
    \hline
    \multirow{2}{*}{Jarak} & \multicolumn{9}{c|}{Hasil Prediksi}    & \multirow{2}{*}{Akurasi} \\ \cline{2-10}
                                & \multicolumn{1}{c|}{1}    & \multicolumn{1}{c|}{2}     & \multicolumn{1}{c|}{3}     & \multicolumn{1}{c|}{4}     & \multicolumn{1}{c|}{5}    & \multicolumn{1}{c|}{6}     & \multicolumn{1}{c|}{7}     & \multicolumn{1}{c|}{8}    & 9    &                          \\ \hline
    3                           & \multicolumn{1}{c|}{2,64} & \multicolumn{1}{c|}{53,61} & \multicolumn{1}{c|}{12,01} & \multicolumn{1}{c|}{8,50}  & \multicolumn{1}{c|}{3,22} & \multicolumn{1}{c|}{11,13} & \multicolumn{1}{c|}{55,08} & \multicolumn{1}{c|}{7,32} & 8,50 & -402,72 \%               \\ \hline
    6                           & \multicolumn{1}{c|}{6,74} & \multicolumn{1}{c|}{5,86}  & \multicolumn{1}{c|}{7,03}  & \multicolumn{1}{c|}{7,62}  & \multicolumn{1}{c|}{7,91} & \multicolumn{1}{c|}{7,62}  & \multicolumn{1}{c|}{7,03}  & \multicolumn{1}{c|}{6,74} & 5,57 & 82,86 \%                 \\ \hline
    9                           & \multicolumn{1}{c|}{9,67} & \multicolumn{1}{c|}{8,79}  & \multicolumn{1}{c|}{9,67}  & \multicolumn{1}{c|}{10,83} & \multicolumn{1}{c|}{7,62} & \multicolumn{1}{c|}{9,08}  & \multicolumn{1}{c|}{9,08}  & \multicolumn{1}{c|}{9,08} & 8,79 & 93,56 \%                 \\ \hline
    \end{tabular}

    \end{table}


Pada jarak 3 meter, didapat rata rata deviasi absolut sekitar 24.72 dengan nilai RMSE 2.72 dan nilai akurasi sekitar -402.72\%. Pada jarak pengujian 6 meter, didapatkan rata rata deviasi absolut sekitar 1.03 dengan nilai RMSE 1.17 dan nilai akurasi sekitar 82.86\%. Sedangkan pada jarak 9 meter, rata rata deviasi absolut adalah sekitar 0.58 dengan nilai RMSE 0.83 dan nilai akurasi sekitar 93.56\%. Maka dapat diambil suatu analisa bahwa kemampuan radar yang telah didesain bekerja sangat baik pada jarak 9 meter dan 6 meter. Sedangkan pada jarak 3 meter, radar yang telah didesain tidak dapat melakukan estimasi jarak objek.


\begin{table}[H]
    \caption{Hasil deteksi Kecepatan}
    \label{tab:estVelocity}
    \begin{tabular}{|c|cccccc|}
        \hline
        \multirow{3}{*}{Kecepatan} & \multicolumn{6}{c|}{Hasil Prediksi}                                                                                                                  \\ \cline{2-7} 
                                   & \multicolumn{3}{c|}{Menjauhi}                                                        & \multicolumn{3}{c|}{Mendekati}                                \\ \cline{2-7} 
                                   & \multicolumn{1}{c|}{1}           & \multicolumn{1}{c|}{2}  & \multicolumn{1}{c|}{3}  & \multicolumn{1}{c|}{1}         & \multicolumn{1}{c|}{2}  & 3  \\ \hline
        5                          & \multicolumn{1}{c|}{ND}          & \multicolumn{1}{c|}{ND} & \multicolumn{1}{c|}{ND} & \multicolumn{1}{c|}{ND}        & \multicolumn{1}{c|}{ND} & ND \\ \hline
        10                         & \multicolumn{1}{c|}{-9.8 KM/H}   & \multicolumn{1}{c|}{ND} & \multicolumn{1}{c|}{ND} & \multicolumn{1}{c|}{10.3 KM/H} & \multicolumn{1}{c|}{ND} & ND \\ \hline
        15                         & \multicolumn{1}{c|}{-18.03 KM/H} & \multicolumn{1}{c|}{ND} & \multicolumn{1}{c|}{ND} & \multicolumn{1}{c|}{ND}        & \multicolumn{1}{c|}{ND} & ND \\ \hline
        20                         & \multicolumn{1}{c|}{ND}          & \multicolumn{1}{c|}{ND} & \multicolumn{1}{c|}{ND} & \multicolumn{1}{c|}{ND}        & \multicolumn{1}{c|}{ND} & ND \\ \hline
        Akurasi                    & \multicolumn{3}{c|}{ND}                                                              & \multicolumn{3}{c|}{ND}                                       \\ \hline
        \end{tabular}
    \end{table}

Pada tabel~\ref{tab:estVelocity} telah dipaparkan beberapa sampel data hasil deteksi kecepatan objek yang bergerak. Bisa dilihat bahwa hasil estimasi kecepatan bisa berjalan, namun tidak dengan kemampuan yang baik. Arah dari gerakan objek yang bergerak juga dapat diamati. Pada saat objek bergerak menjauhi radar, nilai estimasi kecepatan akan negatif, sedangkan saat objek bergerak menjauhi radar, maka nilai estimasi kecepatan menjadi positif.

Karena hasil percobaan yang dapat dianalisis hasilnya hanyalah satu sampel, maka rata rata nilai absolut dan RMSE tidak bisa didapatkan. Sehingga dalam proses penilaian kualitas estimasi, kecepatan yang didapat hanya bisa dinilai dengan menggunakan deviasi absolut. Pada kecepatan 10 km/h mendekati radar, nilai deviasi adalah 0.3. Sedangkan saat objek bergerak menjauhi radar dengan kecepatan 10 km/h, didapatkan hasil negatif yang menunjukkan arah gerak objek, beserta nilai deviasi sekitar 0.2. Pengujian yang memiliki hasil selain dua di atas adalah pengujian 15 km/h yang memiliki nilai deviasi 3.03 dengan penunjuk arah negatif yang benar.

Alasan data hasil pengujian kecepatan sangat sulit untuk didapatkan dan diverifikasi karena penggunaan objek target berupa kendaraan bermotor roda dua. Dengan ukuran tersebut, maka dapat diasumsikan bahwa objek tersebut memiliki \textit{Radar Cross Section} yang tidak terlalu besar, yaitu sekitar $3 m^{2}$.
% Bab 6 : Kesimpulan dan Saran
%---------------------------------------------------------------
\chapter{\babEnam}
%---------------------------------------------------------------
%---------------------------------------------------------------
\section{Kesimpulan}
%---------------------------------------------------------------

Pada bagian ini, kesimpulan dari penelitian akan disajikan. Terutama dalam menjawab rumusan masalah yang diajukan pada penelitian ini, maka kesimpulan yang diperoleh adalah:

\begin{enumerate}
    \item Perancangan sistem radar FMCW dengan GNURadio telah dilakukan dengan menggunakan kemampuan maksimal dari USRP B210. Mulai dari frekuensi sampling 28 MHz bagi \textit{transmit} dan \textit{receive}, lalu lebar \textit{bandwidth} 14 MHz, hingga nilai \textit{internal gain} maksimum untuk mencapai jarak paling jauh yang dapat dicakup. Sistem telah dirancang dan implementasikan dengan baik.
    \item Pengujian sistem radar FMCW telah dilakukan, khususnya untuk melakukan pendeteksi objek, estimasi jarak, dan kecepatan objek. Dengan menggunakan kendaraan roda dua yang memiliki \textit{Radar Cross Section} yang kecil, radar mampu menangkap sinyal pantulan dari objek.
    \item Evaluasi terhadap sistem radar FMCW yang didesain telah dilakukan setelah melakukan pengujian, didapatkan hasil bahwa radar bekerja dengan baik sebagai pendeteksi jarak yaitu pada 6 meter dengan nilai akurasi prediksi 82.86\% dan pada 9 meter dengan nilai akurasi sekitar 93.56\%. Sedangkan sistem radar yang didesain tidak dapat mendeteksi objek pada jarak 3 meter dengan nilai akurasi -402.72\%. Dalam pengujian kecepatan, sistem radar yang didesain dapat melakukan estimasi kecepatan dengan perbandingan perubahan jarak. Dengan begitu maka radar dapat melakukan deteksi objek.
    %\item Dalam penelitian ini, langkah yang jelas telah dipaparkan dalam melakukan perancangan sistem radar FMCW dengan USRP B210 dan GNURadio. Dimulai dengan identifikasi kemampuan USRP, penyesuaian parameter sistem radar yang dirancang, prediksi kemampuan radar, implementasi sistem, dan pengujian radar.
\end{enumerate}

%---------------------------------------------------------------
\section{Saran}
%---------------------------------------------------------------

Dari penelitian yang sudah dilakukan, maka terdapat saran yang bisa di implementasikan pada penelitian selanjuntya.

\begin{enumerate}
    \item Menggunakan objek dengan \textit{Radar cross section} lebih besar untuk memastikan seluruh energi dapat terpantul secara maksimal oleh objek.
    \item Menggunakan perangkat keras lebih mumpuni untuk mengatasi \textit{error overflow} dan \textit{underflow} sehingga pengolahan data secara \textit{real time} dapat dilakukan.
    \item Menggunakan antena dengan \textit{beamwidth} lebih kecil dengan \textit{gain} tinggi untuk memastikan isolasi benar benar terjadi pada gelombang yang di transmisikan.
    \item Menggunakan teknik yang dapat memperbesar lebar pita dari sistem radar FMCW yang terbatas, seperti \textit{frequency hopping} dan \textit{synthetic wide-bandwidth}.
\end{enumerate}

% Daftar Pustaka
%
% Daftar Pustaka 
% 
% 
% Tambahkan pustaka yang digunakan setelah perintah berikut. 
% 
\renewcommand{\section}[2]{}%
\bibliographystyle{ieeetr}
\bibliography{dafpus}
 


%\bibliographystyle{plain}
%\bibliography{ref}

% Lampiran 
\begin{appendix}
	\pagenumbering{gobble}
	\include{markLampiran}
	\setcounter{page}{2}
	%-----------------------------------------------------------------------------%
\addChapter{Lampiran A 1}
\chapter*{Lampiran A}
%-----------------------------------------------------------------------------%

\begin{figure}
    \begin{center}
        \includegraphics[scale=0.08]{pics/Lampiran/Horizonal.jpg}
    \end{center}
\end{figure}

\addChapter{Lampiran A 2}
\begin{figure}
    \begin{center}
        \includegraphics[scale=0.08]{pics/Lampiran/Vertikal.jpg}
    \end{center}
\end{figure}

%-----------------------------------------------------------------------------%
\addChapter{Lampiran B}
\chapter*{Lampiran B}
%-----------------------------------------------------------------------------%

\lstinputlisting[language=Matlab]{pics/Lampiran/ReadTransmitSignal.m}

%-----------------------------------------------------------------------------%
\addChapter{Lampiran C}
\chapter*{Lampiran C}
%-----------------------------------------------------------------------------%

\lstinputlisting[language=Matlab]{pics/Lampiran/postProcessingRange.m}
\end{appendix}

\end{document}