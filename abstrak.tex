
\chapter*{Abstrak}
\vspace*{0.7cm}

Teknologi penginderaan semakin diperlukan di masa mendatang, salah satunya adalah untuk melakukan pendeteksian objek, estimasi kecepatan dan jarak yang digunakan pada alat seperti \textit{speed trap camera} di pinggir jalan untuk mendeteksi kecepatan kendaraan agar tidak melewati batas kecepatan yang sudah ditentukan. Dari berbagai teknik penginderaan terdapat teknologi radar. Radar \textit{Frequency Modulated Continuous Wave} yang populer digunakan merupakan salah satu teknik yang ramai digunakan. Implementasi teknik ini kerap ditemukan di berbagai bidang, mulai dari otomotif hingga kesehatan.\\

Pada penelitian ini, telah dirancang dan diuji radar FMCW berbasis \textit{Software Defined Radio} dengan menggunakan GNURadio untuk melakukan deteksi, estimasi jarak, dan kecepatan dari suatu objek. Spesifikasi sistem radar yang dirancang ada pada frekuensi pembawa 5.8 GHz, dengan bentuk modulasi \textit{Sawtooth}. Implementasi dilakukan dengan satu unit USRP menggunakan antena \textit{log periodic}. Hasil pengujian jarak menunjukkan kualitas yang baik pada jarak 6 meter dengan nilai akurasi prediksi 82.86\% dan pada 9 meter dengan nilai akurasi sekitar 93.56\%, pada pengujian jarak 3 meter sistem radar tidak dapat melakukan prediksi jarak dengan nilai akurasi -402.72\%. Pada pengujian kecepatan hasil tidak menunjukkan kualitas yang baik.



\vspace*{0.2cm}

\noindent \textbf{Kata Kunci}: \textit{FMCW}, \textit{Radar}, \textit{GNURadio}, \textit{USRP}, \textit{Detection}, \textit{Estimation}\\ 

\newpage