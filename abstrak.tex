
\chapter*{Abstrak}
\vspace*{0.7cm}

Teknologi penginderaan semakin diperlukan di masa mendatang, salah satunya adalah untuk melakukan pendeteksian objek, estimasi kecepatan dan jarak yang digunakan pada alat seperti \textit{speed trap camera} di pinggir jalan untuk mendeteksi kecepatan kendaraan agar tidak melewati batas kecepatan yang sudah ditentukan. Dari berbagai teknik penginderaan terdapat teknologi radar. Radar \textit{Frequency Modulated Continuous Wave} yang populer digunakan merupakan salah satu teknik yang ramai digunakan. Implementasi teknik ini kerap ditemukan di berbagai bidang, mulai dari otomotif hingga kesehatan.\\

Pada penelitian ini, telah dirancang radar FMCW berbasis \textit{Software Defined Radio} dengan menggunakan GNURadio untuk melakukan deteksi, estimasi jarak, dan kecepatan dari suatu objek. Spesifikasi sistem radar yang dirancang ada pada frekuensi pembawa 3.1 GHz, dengan bentuk modulasi \textit{Triangular}. Direncanakan bahwa implementasi akan dilakukan dengan dua unit USRP menggunakan antena \textit{log periodic}.



\vspace*{0.2cm}

\noindent \textbf{Kata Kunci}: \textit{FMCW}, \textit{Radar}, \textit{GNURadio}, \textit{USRP}, \textit{Detection}, \textit{Estimation}\\ 

\newpage